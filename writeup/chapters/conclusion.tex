\bbs{Conclusion}

This thesis presents the design of a new style of biologically inspired
controller for controlling revolute joints with antagonistic pneumatic
artificial muscles. First, the literature surrounding the characterization of
the actuators and the methods used to control them is discussed in
\myref{chap:lit_review}. Second, the
design of a prototype for a controller is discussed in
\myref{chap:controller_design}. Third, the implementation of the design in a
synthetic nervous system is discussed in \myref{chap:neuron_design}. The methods
for testing the controller are discussed in \myref{chap:methods}. The results of
the tests are presented in \myref{chap:results}.

The controller implements new features not previously used in synthetic nervous
system design and not previously used for control of joints actuated by
pneumatic artificial muscles. The system has a whole has been shown to offer
increased accuracy in joint position and decreased phase shift. This was
combined with a better internal model of the actuators themselves to more
accurately model the forces and torques applied in order to continue successful
operation even near the maximum output of the actuators. On the other hand, the
successful operation of the controller is dependent on the tuning of a number of
internal parameters that are sensitive to slight overestimates that lead to
oscillation and loss of control of the joint. This has been mitigated by
asymmetric model updates that favor stable error and starting from known stable
estimates; however, this leads to tracking that does not meet the accuracy goals
discussed in the design when the parameters are too conservative.

The implementation of the controller in a synthetic nervous system led to the
design of unique nervous system functional subnetworks; however, some networks
were found to be inconsistent with their standard controller output, leading to
potential deficiencies in observed performance. This would suggest that a more
bionic approach that leverages mathematical models directly for complicated
calculations, such as the relation between pressure and torque for pneumatic
artificial muscles, and the dynamic benefits of a synthetic nervous system for
aspects of the controller that require less precision and benefit from insights
gained from a deeper biological understanding of how animal nervous systems
control joints.

\bbs{Future Work}

The next step for the design of the controller is experimental testing to
characterize performance on actual hardware. During simulation and testing,
many aspects of the controller worked well; on the other hand, some did not
behave as well as expected. On hardware, there is always some variation that may
show that the design choices made were effective and practical solutions.

There is also some potential for improvement of the design of the controller.
Further analysis of the stability of the controller may lead to a better model
update procedure. In particular, algorithms such as an Extended Kalman Filter
are used in many applications across robotics for estimating parameters from
sensor data, but the Kalman Filter algorithm itself was not used because it does
not have an effective way to be implemented within a synthetic nervous system.
New biological research % TODO(buckbaskin): cite that paper with the toroid structure
suggests that animals have groups of neurons that estimate orientation and
% TODO(buckbaskin): and that time and space cells paper thing I found
position in time and space. This research suggests potential approximations for
spatial estimation that may be useful for estimation of other parameters to
emulate or replace the use of a Kalman Filter.

Another area where the neuron controller has room for improvement is
implementation of certain subnetworks that don't effectively implement their
counterparts in the prototype. There were approximations made to match neuron
and synapse behavior at a low level to some of the mathematics; however, taking
a larger system approach with an aim to design neuron connections to emulate
behavior of larger pieces of the system made lead to a more successful design.
The end goal is to design a working controller with similar properties, this
doesn't need to be achieved by direct copying. The testing methods implemented
in this thesis offer a good system for iterating on this kind of design. Test
inputs can be set up once and then higher level component behavior can be
compared quickly and visually.

Overall, the design of the controller presents a new style of design that can be
used to create optimal controllers for actuation systems that are highly
non-linear and tend to perform poorly when simpler controller designs are used.
There is room for improvement, but the methods used in the thesis offer a
framework for improvement and determining the areas with the largest impact.
