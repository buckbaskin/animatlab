\bbs{Conclusion}

This thesis presents the design of a new style of biologically inspired
controller for controlling revolute joints with antagonistic pneumatic
artificial muscles. The
design of a prototype for a controller is discussed in
\myref{chap:controller_design}. The implementation of the design in a
synthetic nervous system is discussed in \myref{chap:neuron_design}. The methods
for testing the controller are discussed in \myref{chap:methods}. The results of
the tests are presented in \myref{chap:results}.

The new controller implements features not previously used in synthetic nervous
system design and not previously used for control of joints actuated by
pneumatic artificial muscles. The system has been shown to offer
increased accuracy in joint position and decreased phase shift in the Python simulation. This increased internal accuracy was
combined with a better internal model of the actuators themselves to more
accurately model the forces and torques applied in order to continue successful
operation even near the maximum output of the actuators. On the other hand, the
successful operation of the new controller is dependent on the tuning of a number of
internal parameters that are sensitive to slight overestimates that lead to
oscillation and loss of control of the joint. The oscillatory behavior has been mitigated by
asymmetric model updates that favor stable error and starting from known stable
estimates; however, these mitigations lead to tracking that does not meet the accuracy goals
discussed in the design when the parameters are too conservative.

The implementation of the controller in a synthetic nervous system led to the
design of unique nervous system functional subnetworks; however, some networks
were found to be inconsistent with their standard controller output, leading to
potential deficiencies in observed performance. This result would suggest that a more
bionic approach that uses mathematical models directly for complicated
calculations, such as the relation between pressure and torque for pneumatic
artificial muscles would offer a more accurate controller. Including a synthetic nervous system based controller enhances the flexibility and adaptability of the controller when there is less precision required and a insights gained from a deeper biological understanding of how animal nervous systems control joints are required.

\bbs{Future Work}

The next step for the design of the controller is experimental testing to
characterize performance on actual hardware. During simulation and testing,
many aspects of the controller worked well; on the other hand, some did not
behave as well as expected. When implemented on hardware, there is always some variation that may
show which design choices made were effective and which required practical solutions.

There is also some potential for improvement of the design of the controller.
Further analysis of the stability of the controller may lead to a better model
update procedure. In particular, algorithms such as an Extended Kalman Filter
are used in many applications across robotics for estimating parameters from
sensor data; however, the Kalman Filter algorithm was not used because the matrix math operations (including an inversion) do
not have a good parallel implementation within a synthetic nervous system.
New biological research % TODO(buckbaskin): cite that paper with the toroid structure
suggests that animals have groups of neurons that estimate orientation and
% TODO(buckbaskin): and that time and space cells paper thing I found
position in time and space. This research suggests potential approximations for
spatial estimation that may be useful for estimation of other parameters to
emulate or replace the use of a Kalman Filter.

Another area where the neuron controller has room for improvement is
implementation of certain subnetworks that do not effectively implement their
counterparts in the prototype. There were approximations made to match neuron
and synapse behavior at a low level to some of the mathematics; however, taking
a larger system approach with the goal of designing neuron connections to emulate
behavior of larger pieces of the system may lead to a more successful design.
The end goal is to design a working controller with similar properties and behaviors to the prototype, this
does not need to be achieved by direct copying of the internal mechanisms. The testing methods implemented
in this thesis offer a good system for iterating on a synthetic nervous system controller design. Test
inputs can be set up once and then higher level component behavior can be
compared quickly and visually.

Overall, the design of the controller presents a new style of design that can be
used to create optimal controllers for actuation systems that are highly
non-linear and tend to perform poorly when simpler controller designs are used.
There is room for improvement, but the methods used in this thesis offer a
framework for improvement and determining the areas of improvement with the largest impact.
