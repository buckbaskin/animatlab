\section{Simulation}

\subsection{Animatlab}

\subsection{Python Numerical Simulation}

\section{Testing and Verification}

\subsection{Neurons}

One of the most important aspects of the functional subnetwork approach is that
individual subnetworks can be tuned, verified and then combined with the larger
network. In order to better tune individual networks, a testing rig was set up
within the larger network to allow for subnetwork verification. The typical
interface neurons are disabled and special test driver neurons (highlighted in
yellow) are enabled. This allows for the driving of custom input signals and
combinations of signals in order to verify the output value and speed.

% TODO(buckbaskin): show an example network with test and not-test inputs. This
% should show the network, the input levels, intermediate levels and output
% levels with a ground truth output level alongside the output levels.

These test frameworks can be rigged at multiple levels to enable testing of
individual units and the integration of units in the style of unit tests and
integration tests from a more traditional software background.

\bbsss{Key Tests}

% TODO(buckbaskin): Show and talk about some specific/important unit tests and 1
% integration test from the network

This section covers in detail important subsections of the network where
explicit testing played a large role in their success. For a complete listing of
the tested networks, please see \myref{app:neuron_tests}.
