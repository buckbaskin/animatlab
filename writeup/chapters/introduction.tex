Walking (especially bipedal) is hard

Animals achieve stability and effective walking through redundancy (ex. 4 legs,
6 legs, etc) and multi-level neuron control via sensory feedback and descending
commands. Studies have shown that animals can still exhibit significant portions
of walking behavior without higher level control from the brain and, in certain
conditions, without sensory feedback.

Animals adapt to their environment through physical structure and compliance as
well as adapting through a learning process in the nervous system. While the
simulation or replication of both the physical and neural systems in their
entirety is a very complex task, lessons can be taken from the neural control of
muscles to better control a walking robot.

\section{Task Overview}

This thesis focuses on control of joints for the canine-inspired
Puppy robot in a walking scenario. The two distinguishing physical features of 
the robot are its pneumatic muscle actuators and a design that leans heavily
on the mechanics and skeleton of canines.

% TODO(buckbaskin): insert figure
Figure of Puppy

% TODO(buckbaskin): finish this thought
Pneumatic actuators are used because...

% TODO(buckbaskin): finish this thought
Canine design is used because...

\section{Why Walking?}

Walking robots, and more specifically quadruped walking robots, offer a number 
of advantages over wheeled robots. First, walking robots can cover much more 
varied terrain compared with wheeled vehicles. Second, quadruped and walking
robots with more than 4 legs offer inherent stability compared with biped robots
on the same terrain via redundancy and the ability to move one limb at a time
while keeping the center of mass over a supported area.


On the other hand, walking robots have disadvantages that mean they are not
necessarily the right option for every situation.

Increased energy usage (energy per meter metric?). This comes from the
requirement that the active control in the robot is required to support the
weight of the robot with motors or muscles instead of having weight statically
supported by the structure in the case of wheels.

% TODO(buckbaskin): write about this more
Lower maximum speed


One option that has been explored as an intermediate step are Whegs. These
designs allow for increased climbing ability on uneven terrain while retaining
many of the advantages of wheels including ease of control (although additional
% TODO(buckbaskin): 1-2 citations for Whegs
work must be done to synchronize a tripod gate) and increased speed.


However, canines and other legged animals have adapted to efficient and fast 
methods for walking using muscles that suggest robotic design inspired by 
biomechanics of a dog can offer insights into making a more efficient and 
faster robot that retains the advantages of legged robots for walking on uneven
% TODO(buckbaskin): incorporate a second reason for walking robots
terrain and remaining stable in varied environments.

\section{Specifications and Important Challenges}

For control of a walking robot, based on previous challenges

% TODO(buckbaskin): break this into subsections and a paragraph for each
% challenge
Challenges:
\begin{itemize}
\item Peak position during hip actuation isn't accurate (5-15 deg) 
\item Delays between sensor data off the robot -> lab view -> Animatlab -> lab
view -> robot cause tracking problems
\item Scapula drives forward motion for front legs in the reverse Z orientation
\item Improve torque profiles to eliminate external support, even close to
maximum activation
\item Better internals and internal understanding of dynamics to match desired
and intended activation
\end{itemize}

% TODO(buckbaskin): break this into subsections and a paragraph for each
% requirement
Requirements:
\begin{itemize}
\item Increase position accuracy to less than 5 degrees from maximum desired position
\item Minimize phase shift between desired and achieved trajectory
\item Decrease torque errors between neuron desired model and executed torques
\end{itemize}

% TODO(buckbaskin): break this into subsections and a paragraph for each
% design goal
Internal Design Goals:
\begin{itemize}
\item Incorporate and ``understanding" of dynamics
\item Optimize current control for future position or, stated another way,
control for the current time/state with delayed sensory information
\end{itemize}

\section{This Work}

\myref{chap:controller_design} 
discusses the design of a new type 
of controller that incorporates the
design goals of understanding the dynamics, projecting forward in time and
optimizing the controlled output based on the understanding of dynamics.

\myref{chap:neuron_design} discusses the implementation of the controller within 
a synthetic nervous system
simulation. This discussion talks about some simplifications and approximations
that were made in the process, as well as areas of the network that are more
complex improvements over the controller design in \Cref{chap:controller_design}.

\myref{chap:methods} discusses the methods use to design, iterate, test and
validate the controller designs.

\myref{chap:results} discusses the results of testing the controllers.

\myref{chap:conclusion} discusses the conclusions and future work.
