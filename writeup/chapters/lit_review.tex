% TODO(buckbaskin): consider being consistent and using only typically instead of usually
\bbs{Puppy}

The canine-inspired robot Puppy was designed to prove the feasibility of using
pneumatic muscles for actuating a quadruped robot. The design focuses on 
imitating the musculoskeletal parameters of an adult greyhound. 
\cite{PuppyDesign} This work is motivated by extensions and potential
improvements to a control method developed by Alex Hunt in
\cite{HuntPhDThesis} and \cite{HuntHindLegWalking}. Among other areas identified
for improvement and future work, the thesis in particular cites accurate torque
profiles as a key improvment for increasing the stability and mobility of the
robot. Related to the accurate torque profiles, the joint peak angles are
accurate to between 5 and 15 degrees, but improving the accuracy further offers
an avenue for more effective walking. One potential source of error in the
hardware system is the delay between a sensor reading, communicating the data to
an offboard computer and then communicating a command back to the robot. The end
result of the current system is that the hind legs can walk effectively;
however, the front legs would trip or otherwise get stuck under the body of the
robot.

\bbs{Similar Robots}

There have been multipled variations of walking robots in 2-legged and 4-legged
configurations. Canines and felines are a common inspiration for these kinds of
robots, with inspiration ranging from very low degree of freedom (single joint
per leg) robots that mimic gait timing to robots that attempt to implement
complete structural and muscular replications of dogs or cats.

\bbss{Canine Robots}

The Big Dog robot made by Boston Dynamics (along with the more recent Spot and
Spot Mini) are quadruped robots built specifically for enhanced all terrain
mobility. According to \cite{BigDog}, less than half of land terrain is
accessible to wheeled or tracked vehicles, so legged robots have the opportunity
to offer unparalleled mobility.

The robot ``Ken" is based on canine kinematics with a focus on designing for
high frequency operation. The design is focused on being lightweight while also
being biomimetic. One key design decision was the use of pneumatic muscles over
electric motors. This facilitates high speed and high frequency operation
without the usual heating and/or cooling issues. Another interesting design
decision for the Ken robot was the incorporation of a contractor muscle at the
hip. Its placement and operation shorten or lengthen the leg mechanism
independent of the flexion or extension of the leg.
\cite{Narioka2012}

\bbsss{Dynamics of Canine Robots}

The dynamics of robots, even those that attempt to accurately mimic biolgoical
mechanisms, can vary significantly in performance from their biological
counterparts. For example, mirroring the standard Z configuration of a canine-
inspired robot can actually increase its efficiency by better utilizing passive
spring elements. This suggests that animals may have adapted gaits and
stimulation patterns for redundant muscles that take better advantage of passive
properties of muscles than are accomodated for in current control designs.
\cite{HindLegMorphology}

Passive elements and the particular properties of a joint play a strong role in
the energy required to actuate the joint and the effective frequency at which it
can operate. In particular, \cite{Na2015} covers the design of a spring
mechanism with a high natural frequency to increase the amplitude of joint
motion during high frequency operation.

\bbss{Other Robots with Pneumatic Muscles}

% TODO(buckbaskin): insert a photo of pneumatic air muscles, one relaxed and one pressurized

% TODO(buckbaskin): cite andirpikolos survey
In this survey, many biorobotic applications for pneumatic muscles are presented.
Additionally, the devices are used for medical applications where the actuators
inherent compliance, lighter weight and higher power output are also important. 

The Pneupard robot is a feline-inspired robot based on a cheetah's
musculoskeletal design. It is actuated using pnematic muscles to control each 3
DOF leg. To control the leg, a rules-based controller is used to switch between
states where muscle activations in an actual cat are mapped to pressures in each
muscle. This was aided by taking a biomimetic approach to determining the lever
arms by which the pneumatic muscles actuate. The results showed that the simple
control system remained effective even in the face of obstacles in the robot
path due to the compliant biomechanical design.
\cite{Pneupard2013}

In \cite{Wait2014}, the authors show that a simpler controller design can be
used on a quadruped robot with biologically inspired design to achieve a highly
power dense design that walks effectively. The control approach focuses on
a high-impedance control scheme for the stance phase and a second state
that takes advantage of pneumatic actuators's compliance during the swing phase.
This design does not use pneumatic muscles. Instead, the authors use pneumatic
cylinders to make the design more compact by using a device that can apply both
pushing and pulling forces on a joint. 

\bbs{Pneumatic Actuators}

Pneumatic artificial muscles typically consist of a high pressure source, a
flexible polymer section and an exhaust valve. The muscle contracts in length
when the internal pressure is higher than the surrounding environment. During
this contraction, the muscle applies a force. One limitation of pneumatic
muscles is that they are typically unable to apply force during relaxation in
the same way a rigid pneumatic cylinder can. Instead, they are typically used in
antagonistic pairs. The nonlinear nature of the muscles combined with the
antagonistic design means that, in practice, simple linear controllers are
ineffective. This means that more complex controllers are required; however,
this also means that the controller design can be more detailed and take
advantage of the unique properties of two antagonistic nonlinear actuators.

Pneumatic actuators less common in robotics compared with electric motors;
however,
they are often result in a lighter system for controling a joint than an
electric motor. There are tradeoffs to each, but pneumatic actuators offer a
higher power to weight ratio for smaller displacement at low velocity and are
often a significant improvment over electric motors at higher velocities.
\cite{Tavakoli2008} This suggests that they are well suited for applications in
biological walking robots where an increased power to weight ratio increases
the efficiency of the walking mechanism, especially at high joint velocities
found in a stepping motion.

\bbss{Characterizing Pneumatic Actuators}

In \cite{Situm2008}, the authors analyze gas flow properties for fluidic muscles
and experimentally quantifying their performance in an antagonisitc coupling
through a pulley. One of the key conclusions motivating future work is the
demonstration for a single joint with most external factors removed, the
proportional-integration control as implemented was unable to fully correct for
errors even in a desired trajectory composed of slow steps to a constant value
where the control system had time to reach a set actuation pressure.

Pneumatic Actuators are used on the Puppy robot. One of the key
benefits of using pneumatic muscles in a biologically-inspired robot design is
that the muscles have similar properties to the biological muscles that control
canines. \cite{Tavakoli2008}
During static analysis, the muscles demonstrated a non-linear response to
changing length and the ability to apply a force. This relationship was
quantified and generalized in \cite{HuntPMuscles}. This research showed that
there is a relationship between the strain on the muscle and the pressure
required to support a fixed load that can be approximated by a shifted tangent
function. Using test and validation sets, the model was fit to data collected
from existing actuators on Puppy. Further, the paper goes on to show that the
model helps derive a controller for semi-static motion of joints. 
\cite{HuntPMuscles}

Other research with pneumatic muscles provides a model for incorporating the
dynamic properties of pneumatic actuators to allow for a more fine tuned control
process. In \cite{DynamicPMuscles}, the authors use dynamic application and
removal of load to measure the effective spring constant of the muscles and the
effective damping constant. The spring constant was shown to be correlated with
pressure, with a small constant spring effect at no pressure from the muscle
material itself. The damping coefficient was shown to be positively correlated
with pressure during contraction and weakly negatively correlated with pressure
during relaxation. The work suggests that this damping comes from intenal
friction in the actuator design. Additionally, the damping due to friction
effects in the input and exhaust lines is shown to be minimal compared to
internal actuator effects. \cite{DynamicPMuscles}

\cite{einstein}

\bbs{Controllers}

% 2
% TODO(buckbaskin): cite MLS94 Robits 1 textbook
Joint position controllers can be treated as a function mapping the current
state (usually position, velocity and acceleration) to a torque to match the 
desired trajectory, expressed as position but often constrained to be a twice
differentiable function.

\begin{equation}
M(\theta) \ddot{\theta} + C(\theta, \dot{\theta}) \dot{\theta} + N(\theta, \dot{\theta}) = \tau
\end{equation}

If the robot starts at its desired initial state and the model of the robot is
perfectly known, the robot will track the exact trajectory. Feedback in the 
form of sensory input is used to correct for starting error or error in the 
forumulation of the exact model.

One common formulation for control is a proportional-derivative controller, 
where torque is chosen as the following:

\begin{equation}
\tau = -K_{v} \dot{e} - K_{p} e
\end{equation}

where

\begin{equation}
e = \theta - \theta_{d}
\end{equation}

This results in a control system that is stable, but it doesn't not converge to
exact tracking for most real systems. A common modification for this is to add
an integral term that accounts for steady state error; however, this can lead to
windup or other issues.

One general method for overcoming these limitations is to use a model of the 
system to perform what is known as computed torque control, where the actuator
applies enough torque to account for the $C(\theta, \dot{\theta})$ and
$N(\theta, \dot{\theta})$ terms and then applies the desired acceleration. This feedforward term is written as follows:

\begin{equation}
\tau = M(\theta) \ddot{\theta_{d}} + C(\theta, \dot{\theta}) \dot{\theta} + N(\theta, \dot{\theta})
\end{equation}

To correct for intitial error or otherwise, additional proportional terms are
added to the acceleration term:

\begin{equation}
\tau = M(\theta) (\ddot{\theta_{d}} - K_{v} \dot{e} - K_{p} e) + C(\theta, \dot{\theta}) \dot{\theta} + N(\theta, \dot{\theta})
\end{equation}

These additional terms are known as the feedback component. Since these 
corrections are linear in a system that is linearized by the feedforward term,
values for the gains can be easily chosen to make the system stable and converge
to exactly following the desired trajectory (exponentially stable). This system 
of control is not always practical because it requires a good knowledge of the 
system under control and a model that is well fit to the system. In practice, 
there are many different ways to achieve effective control for a rotational 
joint.

\bbss{Joint Controllers}

In \cite{EventBasedWalking}, a new controller paradigm is proposed where control
is done by switching between known phases in a state machine based on sensory
events to generate an adaptable trajectory similar to purely time-based
trajectories where stability was achieved by a separate stability adjustment
system. This paper focuses on control of a biped robot; however, the lessons can
be applied to a quadruped.

Event-based control through reflexes was also proposed as a way for additional
improvement of the existing synthetic nervous system in \cite{HuntPhDThesis}.
The thesis suggest that a particular a reflex known as the elevator reflex,
where the paw strikes an
obstacle mid-swing, could be used as the design inspiration for an internal recovery
mechanism to prevent tripping.

\bbss{Controllers for Pneumatic Muscles}

This paper talks about using Active Force Control (pretty much what I'm doing)
to control a pneumatic muscle for a linear joint.
% 3
% TODO(buckbaskin): justify why my work is more interesting (in neurons, stable estimation of parameters)
% TODO(buckbaskin): find a good paper to summarize active force control
\cite{Jahanabadi2009}

This paper talks about using model based control (sounds similar to what I'm 
doing) to control a pneumatic muscle joint based on an acceleration trajectory.
% 4
\cite{Wang2013}

\bbs{Similar Neurons}

% 5
% TODO(buckbaskin): add a introduction motivating the use of neurons for controllers

\bbss{CPGs}

Many neuron control systems for oscilating joints incorporate central pattern
generators (CPGs).
% TODO(buckbaskin): including (cite more papers that use CPGs)
In their most reduced form, CPGs are a small bundle of
neurons that oscillate continually without requiring outside stimulation. In
practice, they are often connected to muscles through pattern formation or other
layers to convert the CPG cycle into muscle activation.
% 6
% TODO(buckbaskin): include more contents of CPGReview
\cite{CPGReview}

Multi-Level CPG Organization: low level PID/control loop with higher level CPG
% 7
\cite{MultiLevelCPG}

In insects, CPGs have been shown to combine with peripheral sensory feedback and
descending signals to coordinate walking motion between multiple joints and
limbs.
% 8
CPGs for baseline, sensors/reflexes to reset \cite{SixLeggedWalking}

\bbss{Neuron Design}

Neuron-based controllers are often designed in a data-driven way, where the
neuron system is treated as a black box. Recent work, collected in 
\cite{NickFunctionalSubnetwork}, shows that a more detailed neuron model can be
used to build up small subnetworks of neurons that perform a specific task.
These subnetworks can be recursively combined to form larger and more intricate
networks that form a controller. This approach allows for the integration of
existing work in controller design with biological inspiration to increase the
effectiveness of the controller when controlling a biologically inspired robot.
This approach also allows for rapid testing and developing new hypothesis on how
biological systems work in a way that typical controller designs do not.
