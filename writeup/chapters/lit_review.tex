\bbs{Puppy}

The canine-inspired robot Puppy was designed to prove the feasibility of using
pneumatic muscles for actuating a quadruped robot. The design focuses on 
imitating the musculoskeletal parameters of an adult greyhound. 
\cite{PuppyDesign}

\bbs{Similar Robots}

There have been multipled variations of walking robots in 2-legged and 4-legged
configurations. Canines and felines are a common inspiration for these kinds of
robots, with inspiration ranging from very low degree of freedom (single joint
per leg) robots that mimic gait timing to robots that attempt to implement
complete structural and muscular replications of dogs or cats.

\bbss{Canine Robots}

\bbsss{Example Robots}

\bbsss{Dynamics}

The dynamics of robots, even those that attempt to accurately mimic biolgoical
mechanisms, can vary significantly in performance from their biological
counterparts. For example, mirroring the standard Z configuration of a canine-
inspired robot can actually increase its efficiency by better utilizing passive
spring elements. This suggests that animals may have adapted gaits and
stimulation patterns for redundant muscles that take better advantage of passive
properties of muscles than are accomodated for in current control designs.
\cite{HindLegMorphology}

Passive elements and the particular properties of a joint play a stron role in
the energy required to actuate the joint and the effective frequency at which it
can operate. \cite{Na2015}

\bbss{Robots with Pneumatic Muscles}

% TODO(buckbaskin): find survey of other robots that use pneumatic muscles

Cat with multiple redundant and antagonistic actuators
\cite{Rosendo2013}

Canine robot (cheetah-like?) with a rules based controller that switched between
pressures tied to neuron activations
\cite{Pneupard2013}

Quadruped robot, biologically inspired but not canine biomimetic \cite{Wait2014}



\bbs{Pneumatic Actuators}

Compare air muscles to electric motors
\cite{Tavakoli2008}

\bbss{Characterizing Pneumatic Actuators}

Characterize pneumatic actuators for use in a (linear?) controller
\cite{Situm2008}

Pneumatic Actuators are used on the Puppy robot. They are the primary form of
actuation, replacing electric motors or hydraulic actuation. One of the key
benefits of using pneumatic muscles in a biologically-inspired robot design is
that the muscles have similar properties to the biological muscles that control
canines. 
% TODO(buckbaskin): find a biology paper that covers the strain-effort relationship for muscles
During static analysis, the muscles demonstrated a non-linear response to
changing length and the ability to apply a force. This relationship was
quantified and generalized in \cite{HuntPMuscles}. This research showed that
there is a relationship between the strain on the muscle and the pressure
required to support a fixed load that can be approximated by a shifted tangent
function. Using test and validation sets, the model was fit to data collected
from existing actuators on Puppy. Further, the paper goes on to show that the
model helps derive a controller for semi-static motion of joints. 
\cite{HuntPMuscles}

Other research with pneumatic muscles provides a model for incorporating the
dynamic properties of pneumatic actuators to allow for a more fine tuned control
process. In \cite{DynamicPMuscles}, the authors use dynamic application and
removal of load to measure the effective spring constant of the muscles and the
effective damping constant. The spring constant was shown to be correlated with
pressure, with a small constant spring effect at no pressure from the muscle
material itself. The damping coefficient was shown to be positively correlated
with pressure during contraction and weakly negatively correlated with pressure
during relaxation. The work suggests that this damping comes from intenal
friction in the actuator design. Additionally, the damping due to friction
effects in the input and exhaust lines is shown to be minimal compared to
internal actuator effects. \cite{DynamicPMuscles}

\cite{einstein}

\bbs{Controllers}

\bbss{Overview from Robotics 1}

\bbss{Joint Controllers}

In \cite{EventBasedWalking}, a new controller paradigm is proposed where control
is done by switching between known phases in a state machine based on sensory
events to generate an adaptable trajectory similar to purely time-based
trajectories where stability was achieved by a separate stability adjustment
system. This paper focuses on control of a biped robot; however, the lessons can
be applied to a quadruped.

\bbss{Controllers for Pneumatic Muscles}

This paper talks about using Active Force Control (pretty much what I'm doing)
to control a pneumatic muscle for a linear joint.
% TODO(buckbaskin): justify why my work is more interesting (in neurons, stable estimation of parameters)
% TODO(buckbaskin): find a good paper to summarize active force control
\cite{Jahanabadi2009}

\bbs{Similar Neurons}

\bbss{CPGs}

Many neuron control systems for oscilating joints incorporate central pattern
generators (CPGs). In their most reduced form, CPGs are a small bundle of
neurons that oscillate continually without requiring outside stimulation. In
practice, they are often connected to muscles through pattern formation or other
layers to convert the CPG cycle into muscle activation.
% TODO(buckbaskin): include more contents of CPGReview
\cite{CPGReview}

Multi-Level CPG Organization: low level PID/control loop with higher level CPG
\cite{MultiLevelCPG}

In insects, CPGs have been shown to combine with peripheral sensory feedback and
descending signals to coordinate walking motion between multiple joints and
limbs.
CPGs for baseline, sensors/reflexes to reset \cite{SixLeggedWalking}

\bbss{Neuron Design}
