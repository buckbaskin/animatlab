\bbs{Simulation}

Antagonistic actuators were simulated at various pressures to better understand
their behavior using \github{stability/constant\_pressure.py}. In \myref{fig:AntagonisticPressureTorque}, the extension
actuator was set to a constant 500 kPa. The pressure in the flexion actuator was
varied, and the torque was plotted for a range of actuator angles. This model of
torque generated was used in ongoing work to help better understand joint
dynamics and improve the performance of the controllers.

\begin{figure}
\centering
\includegraphics[height=3.5in]{results/Pos_v_AntagTorque}
\caption{Relation between joint position, actuator pressures and
torque applied. Extension actuator pressure was set to 500 kPa and flexion was
set as indicated.}
\label{fig:AntagonisticPressureTorque}
\end{figure}

During the characterization of the joint, simulations were run to determine the
dominant effects between inertia, damping and static loads using \github{stability/max\_torque.py}. The relative 
magnitudes of the acceleration, velocity and position are shown in 
\myref{fig:MaxTorque}, along with the
torque. For the simulated joint, the dominant effect was the inertia; however,
for a robot carrying varying loads, the inertia will not always be the dominant factor. The 
maximum torque required to execute the known trajectory is also later used as a
metric for measuring the efficiency of the controller. A controller that is too
aggressive for the trajectory (ex. too high gains on a proportional controller)
will request larger than needed torques during trajectory execution. An ideal
controller will always follow the trajectory with the exact torque needed during steady state operation.

\begin{figure}
\centering
\includegraphics[height=3.5in]{results/Max_Torque}
\caption{Comparing the effects of inertia, damping and static loads. Torque follows acceleration, which suggests that inertia dominates for this simulation}
\label{fig:MaxTorque}
\end{figure}

\bbs{Python Controller}

The prototype controller written in Python was used to demonstrate the stability
and tracking capabilities of the general algorithm. The prototype controller also was used to find a
suitable simplification of the internal model that was sufficiently accurate for
optimization but had minimal parameters to tune.

\bbss{Simplified Controller}

The simplified controller demonstrated effective control once the weight updates
were combined with a suitable starting state. \myref{fig:SimplifiedTracking}
demonstrates the progression from original state to a stable and accurate
control of joint position. The blue line is the desired trajectory, with purple
used to highlight error bounds of $\pm$ 1 degree. Orange indicates the actual
position of the joint, and the green line indicates the internal state
estimation.

\begin{figure}
\centering
\includegraphics[height=4in]{results/State_Estimation}
\caption{Tracking improves with improved state estimation and internal model
updates}
\label{fig:SimplifiedTracking}
\end{figure}

\bbss{Static Controller}

During testing, the need for an internal model update was verified through 
tests where the optimization loop was applied with variations on 
parameters using \github{stability/simple\_mass\_model.py}. One example was varying mass. For correct values, such as 
\myref{fig:StateEstimationPerfect}, the system worked within tolerances. When 
mass was underestimated (see \myref{fig:StateEstimationLowMass}), the tracking 
performance degrades to smoothly fail required tracking accuracy. On the other 
hand, over-estimating the inertia caused aggressive oscillation and underdamping (see \myref{fig:StateEstimationHighMass}). Based on these tests, the assumption 
that the internal model needs to update was validated.

\begin{figure}
\centering
\includegraphics[height=4in]{results/State_Estimation_Perfect}
\caption{Accurate tracking with a good internal estimation of mass}
\label{fig:StateEstimationPerfect}
\end{figure}

\begin{figure}
\centering
\includegraphics[height=4in]{results/State_Estimation_LowMass}
\caption{Overdamped tracking with an internal underestimation of mass}
\label{fig:StateEstimationLowMass}
\end{figure}

\begin{figure}
\centering
\includegraphics[height=4in]{results/State_Estimation_HighMass}
\caption{Underdamped failure with an internal overestimation of mass}
\label{fig:StateEstimationHighMass}
\end{figure}

\bbss{Simple Parameter Estimation}

The original model for parameter updates was to calculate the sign of the update
and make a uniform incremental change in that direction. In practice, the constant uniform update
resulted in near-perfect tracking; however, the weight updates themselves were continually varying instead of stabilizing to a true value. See \myref{fig:StateUpdateSimple} and \github{stability/simple\_mass\_model.py}.

\begin{figure}
\centering
\includegraphics[height=7.5in]{results/State_Model_SimpleUpdate}
\caption{Successful tracking with simple parameter update}
\label{fig:StateUpdateSimple}
\end{figure}

\bbs{Neuron Controller}

Animatlab does not offer a good model of pneumatic muscles for use in closed loop testing. In place of this testing, the Python simulation was used to simulate the behavior of the pneumatic actuators and the neuron model was tuned to try and match the output of the Python controller from the small component subnetworks up to the overall controller behavior. The standard for a good match was less than 10 percent error across the simulated range of neuron inputs. For neurons with an active range between -60 mV and -40 mV, the determination for a good network is an average error of less than 2 mV. 

\bbss{Test Results}

Test results are generated based on data tools in \github{PuppyNeuronPlayground}. Post processing and visualization is done with scripts found in \github{writeup/scripts}.

\begin{figure}
\centering
\includegraphics[height=2.25in]{results/TestVelPosWide}
\caption{Testing positive velocity estimation}
\label{fig:TestVelPos}
\end{figure}

\begin{figure}
\centering
\includegraphics[height=2.25in]{results/TestVelNegWide}
\caption{Testing negative velocity estimation}
\label{fig:TestVelNeg}
\end{figure}

\begin{figure}
\centering
\includegraphics[height=3.5in]{results/New_T2A}
\caption{Testing conversion from torque to acceleration with varying link inertia. Mean error is 0.6 mV}
\label{fig:TestAccelInertia}
\end{figure}

\begin{figure}
\centering
\includegraphics[height=3.5in]{results/New_T2A_2} % TODO(buckbaskin): complete this thought with a picture and data
\caption{Testing conversion from torque to acceleration, with varying joint damping. Mean error is 0.7 mV}
\label{fig:TestAccelDamping}
\end{figure}

\begin{figure}
\centering
\includegraphics[height=3.5in]{results/New_P2T} % TODO(buckbaskin): complete this thought with a picture and data
\caption{Testing conversion from actuator pressure to torque. Mean error is 1.4 mV}
\label{fig:TestP2T}
\end{figure}

% TODO(buckbaskin): does it make sense to add a damping graphic in here?

\begin{figure}
\centering
\includegraphics[height=3.5in]{results/New_TO}
\caption{Estimating positive torque required to reach a desired position. Mean error is 0.3 mV}
\label{fig:TestTorqueOptimizationPos}
\end{figure}

\begin{figure}
\centering
\includegraphics[height=3.5in]{results/New_TO_2}
\caption{Testing negative torque required to reach a desired position. Mean error is 2.7 mV}
\label{fig:TestTorqueOptimizationNeg}
\end{figure}

\begin{figure}
\centering
\includegraphics[height=3.5in]{results/New_T2P}
\caption{Estimated extension pressure from desired torque. Mean error is 1.6 mV}
\label{fig:TestT2PPos}
\end{figure}

\begin{figure}
\centering
\includegraphics[height=2.25in]{results/TestSystemCPosWide}
\caption{Positive neuron and actual damping factor updates}
\label{fig:TestSystemCPos}
\end{figure}

\begin{figure}
\centering
\includegraphics[height=2.25in]{results/TestSystemCNegWide}
\caption{Negative neuron and actual damping factor updates}
\label{fig:TestSystemCNeg}
\end{figure}

\begin{figure}
\centering
\includegraphics[height=2.25in]{results/TestSystemNPosWide}
\caption{Positive neuron and actual load factor updates}
\label{fig:TestSystemNPos}
\end{figure}

\begin{figure}
\centering
\includegraphics[height=2.25in]{results/TestSystemNNegWide}
\caption{Negative neuron and actual load factor updates}
\label{fig:TestSystemNNeg}
\end{figure}

\bbsss{Sensor Fusion}

Two key estimates are made within the sensor fusion subnetwork: velocity and
acceleration. The synthetic nervous system controller was tested against sine waves of known
frequencies. The positive and negative estimated are compared to the ground truth in \myref{fig:TestVelPos} and \myref{fig:TestVelNeg}.
Within the neuron network, acceleration is calculated primarily from actuator
pressures. For testing, this network was split into two components: estimating
torque based on actuator pressure (\myref{fig:TestP2T}) and estimating acceleration from torque. The
output of the neuron model is compared to the expected output in
\myref{fig:TestAccelInertia} and \myref{fig:TestAccelDamping}. All 3 tests met the standard, with
mean errors of 1.4 mV, 0.6 mV and 0.7 mV.

% TODO(buckbaskin): check figures in here

\bbsss{Torque Optimization}

The torque optimization network was tested as a complete network. Taking the entire network as a whole, the inputs (position, velocity and desired position) are driven to input values and the outputs are compared with the torque predictions of the prototype controller. This is a higher level test designed to comprehensively test the connectivity between smaller subnetworks examined in other tests. The results of the tests are shown in \myref{fig:TestTorqueOptimizationPos} and \myref{fig:TestTorqueOptimizationNeg}. The positive network is quite accurate with a mean error of 0.3 mV; however,
the error for the negative torque prediction was higher, at 2.7 mV. Due to the difference between positive and negative results, this suggests that there may be significant enough difference between signal transfer and inverter synapses such that, over the course of a longer loop, the accuracy diverges.

The components for converting the torque to
acceleration also were tested
separately. See \myref{fig:TestAccelInertia} and \myref{fig:TestAccelDamping}. As discussed in the Sensor Fusion results, the acceleration calculations were accurate to within 1 mV on average across the range of data points.

The components for converting the torque to pressure also were tested
separately. See \myref{fig:TestT2PPos}.
The test results indicated an average error of about 1.6 mV across the entire
working range of the joint.

\bbss{System Modeling}

The system modeling network was tested as a complete network because both the
damping and load factors are updated from the same $\lambda$ value. See 
\myref{fig:TestSystemCPos}, \myref{fig:TestSystemCNeg},
\myref{fig:TestSystemNPos} and \myref{fig:TestSystemNNeg}.

\bbss{Discussion}
\label{chap:discussion}

% TODO(buckbaskin): edit discussion to reflect "new" data

In this thesis, we propose a new system for controlling joints actuated with
pneumatic artificial muscles. The design focuses on improving controller
performance through improved sensor processing, an internal optimization step
and an observer that continually updates the internal physics model to improve
the optimization step. This design was implemented as part of a synthetic
nervous system controller, with modifications made to the initial design to aid
in conversion to a neuron and synapse model and to take advantage of the benefits of the
synthetic nervous system approach.

\bbsss{Sensor Fusion}
\bbssss{Velocity}

The velocity network in the sensor fusion algorithm varies the most from the
reference data (see \myref{fig:TestVelPos} and \myref{fig:TestVelNeg}). This is a known phenomena addressed in
\cite{NickFunctionalSubnetwork}. This variance is observed in two quantities,
the phase shift and the magnitude.

In the velocity network, the magnitude of the positive and negative
velocities varied quite widely compared with the reference, especially at
higher gains. During the tuning of the network, a balance was chosen between
the maximum phase lag, minimized by decreasing the time constant of the slower
neuron, and increasing the synaptic gain (and therefore the observed difference between
positive and negative velocities).

This error suggests that there is an underlying error in the way that positive
and negative operations are computed; however, through individual tuning the
synapses appear to be relatively accurate. The use of the derivative network
varies in each application across different subnetworks. The engineering
solution for the correct balance of accuracy, gain and phase varies. This
variance leaves open the opportunity for biology to inform a single true correct
solution or a more complex derivative network that can independently select for
gain and phase.

\bbssss{Acceleration}

The acceleration network converges to a near-correct solution.
This convergence demonstrates that the feedback loop concept can be effective; however, the process takes a relatively long, fixed amount of time to
converge from solutions that are a significant step away. This time delay may lead to
issues with rapidly changing pressures on a hardware system. One potential
solution is to modify the time constants of the network
elements within the loop, in particular the integration network, to allow the
internal loop(s) to run faster than the rest of the network
so the outputs of the internal loop appear to converge to the correct value on the same time scale as the
rest of the network. 

This feedback loop design pattern behavior matches the behavior of a synthetic nervous system implementation of adapting force walking in \cite{LegLocal}. This feedback pattern is effective for local learning of control adaptations in synthetic nervous systems and also represents an adaptation of feedback behavior observed in insects \cite{LegLocal}.

\bbsss{Torque Optimization}

In many cases the torque optimization network converges to the
correct solution; however, like the feedback in the acceleration network, the torque optimization process can
take a relatively long time to converge to the correct solution from discontinuous starting
values. Due to the integrator included in the torque optimization network, the starting conditions
for the first test case are based on other tests and did not show consistent
behavior.

The concept of the torque optimization network appears to be effective; however,
its implementation in neurons suffers from both attempting to converge to a
solution quickly and approximating a physical simulation internally. The torque optimization network
uses relatively low time constants and high gains. This can cause overshoot
(seen in \myref{fig:TestTorqueOptimizationPos}) where the
estimated torque is off the charts for either large positive or negative swings.
The high gains also reduce accuracy by amplifying small errors made during the
approximation of the ``physics" implementation. This behavior replicates
observations made during the development of the prototype controller in code,
where an attempt to approximate the physics of the joint in a single step was
inaccurate. The solution to the simulation error was to make multiple iterations by subdividing
the simulated time so that a simpler physics model would have a closer fit to
the actual dynamics. The iteration in the prototype controller suggests that a longer string of neurons would be
necessary within the loop to project dynamics forward; however, increasing the length of the feedback loop may accentuate existing delays when interfaced to hardware.

\bbssss{Acceleration from Torque}

The 3 stage network for estimating the acceleration from the torque applied to
the system appears to work well. There are two test cases, between 7 and 9
seconds that were not significant when tested at the resolution that the
network incorporates (shown in \myref{fig:TestT2APos} and \myref{fig:TestT2ANeg}). Otherwise, the acceleration is almost always correct.
Given its intended uses, the network is sufficiently accurate and suggests that
relatively small sets of neurons can provide a representation of the physics of
a biological system. This would suggest that an animal's nervous system can understand the dynamics of its environment and its body.

\bbssss{Estimating Pressures from Torques}

The process for calculating pressures from torques is accurate in some
cases and inaccurate in many others. This varying accuracy suggests that the physical
model of the pneumatic artificial muscles proposed in \cite{HuntPMuscles}
does not convert well to a relatively small network of neurons. A future controller iteration may elect to move the
calculation of this conversion out of neurons and
to LabView other other code used to interface between synthetic neurons and
hardware.

\bbsss{System Model}

The network for predicting the weight update is very accurate and suggests that
the engineered process of estimating the modeling errors of a synthetic neuron
network within the network itself is both feasible and an effective process for
the continual improvement of a controller applied to a particular hardware
joint under control.

This synthetic nervous system application also is generally more suitable to neuron networks than say,
the calculation of the conversion from pressures to torques following a well
prescribed algorithm involving a shifted tangent function. The exact magnitude
of the update of the weights, especially at a reduced gain, is not as important
as the relative magnitude and sign of each update. This characteristic means that the
implementation of an arithmetic operation or dynamic calculation by a synapse of
a collection of synapses is tolerant to small variances from the expected
behavior.
