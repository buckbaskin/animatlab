The controller design had 3 major goals derived from
requirements and challenges faced with a neural implementation in
\cite{HuntPhDThesis}. See \myref{chap:introduction} for more details.

\begin{enumerate}
\item Increase position accuracy to less than 5 degrees from maximum desired
position
\item Minimize phase shift between desired and achieved trajectory
\item Decrease torque errors between neuron desired model and executed torques
\end{enumerate}

Requirement 2 is fulfilled in
the controller through sensor fusion to estimate state at the time of sensor
readings (assumed to be ``current" time) and then using an internal model of the
joint dynamics to predict the evolution of the joint's position and velocity
given a particular controlling torque. From there, an internal optimization loop
can run to identify a goal torque with sufficient resolution.

An accurate internal model will help meet all 3 requirements. In particular, an
internal model directly solves Requirement 3 by matching controlled torques to
pressures accounting for the pneumatic muscle dynamics. The internal model
designed into the control is designed to adapt to the actual observed
characteristics of the joint, allowing for changing dynamics over time (for
example, adding or removing weight from the robot, degrading pneumatic muscles
or changing positions of other limbs),
adapting to different joints with the same controller and increasing the
robustness of the controller to variations from the original estimated
properties of the robot.

Requirement 1 is met by a combination of the design decisions above and
constitutes a metric by which to measure the effectiveness of the new design
relative to old designs, regardless of implementation.

\section{Sensor Fusion}

Within the controller, state estimation is done in a relatively simplified
manner. This trades off accuracy for computation time, where the estimation was
refined to run sufficiently accurately that the rest of the system was robust to
small estimation errors.

\subsection{Pneumatic Actuator Simulation/Simplification}

The pneumatic actuators and their existing pressure controller were simulated 
at varying degrees of fidelity ...

% TODO(buckbaskin): finish this thought based on code

\subsection{Centered Divided Difference}

The velocity was estimated with a fixed window of the last 3 position readings.
From there, a centered divided difference formula was used to estimate the
slope and estimate velocity. This was found to be more accurate than a forward
divided difference method only incorporating two points in practice when run
against simulated data.

% TODO(buckbaskin): write the two equations

% TODO(buckbaskin): show a plot with the two different estimations side by side
% to show error

\section{Optimizing Torque Control}

Once the current state of the robot was determined, iteratively optimizing the
desired torque was relatively simple. First, bounds were selected from estimates
of the maximum and minimum (maximum negative) torque that can be applied to the
joint and a forward projection mechanism based on the internal physics model was
used to estimate the resulting position. From there, a binary search procedure
can be used to repeatedly shrink the bounds around the optimal torque. The final
step was to convert the desired torque into a pressure control.

\subsection{Forward Projection of State}

Given the state estimation, project forward.

Also, can use bounds on estimated state, say high and low velocity, to project
forward through multiple iterations, and optimize for the torque that, given the
bounds, is most likely to produce the best result. This work is unnecessary for
a well implemented state estimation system, but is helpful for increasing
robustness to potential sensor reading errors or estimation errors.

% TODO(buckbaskin): complete this thought

\subsection{Torque to Pressure}

% TODO(buckbaskin): complete these two steps
Show the math

Also, talk about designing the desired pressure around the knowledge of a
bang-bang controller under the hood for more accurate pressure control

\section{System Modeling}

Parameter weight updates

\subsection{Actual Error Calculation}

% TODO(buckbaskin): adapt existing work on how the error update should be done
This comes from a write up that I've already done

\subsection{Evaluation of asymmetric error concerns}o

% TODO(buckbaskin): use some sort of math to show that the over/underestimation
% of damping, etc. can lead to underdamped or overdamped conditions. Prefer
% overdamped control
