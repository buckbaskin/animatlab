The controller design had 3 major goals derived from
requirements and challenges faced with a neural implementation in
\cite{HuntPhDThesis}. See \myref{chap:introduction} for more details.

\begin{enumerate}
\item Decrease torque errors between neuron desired model and executed torques
\item Minimize phase shift between desired and achieved trajectory
\item Increase position accuracy to less than 5 degrees from maximum desired
position
\end{enumerate}

Requirement 1 is fulfilled in
the controller through sensor fusion to estimate state at the time of sensor
readings (assumed to be ``current" time) and then using an internal model of the
joint dynamics to predict the evolution of the joint's position and velocity
given a particular controlling torque. From there, an internal optimization loop
can run to identify a goal torque with sufficient resolution.

An accurate internal model will help meet all 3 requirements. In particular, an
internal model directly solves Requirement 1 by matching controlled torques to
pressures accounting for the pneumatic muscle dynamics. The internal model
designed into the control is designed to adapt to the actual observed
characteristics of the joint, allowing for changing dynamics over time (for
example, adding or removing weight from the robot, degrading pneumatic muscles
or changing positions of other limbs),
adapting to different joints with the same controller and increasing the
robustness of the controller to variations from the original estimated
properties of the robot. This also avoids a phase shift for Requirement 2. 

Requirement 3 is met by a combination of the design decisions above and
constitutes a metric by which to measure the effectiveness of the new design
relative to old designs, regardless of implementation.

\section{Sensor Fusion}

Within the controller, state estimation is done in a relatively simplified
manner. The design trades off accuracy for computation time, where the 
estimation was designed to have sufficiently small error that it wouldn't 
break the functionality of the rest of the system. An additional consideration
was the ability to implement the algorithm in a neuron system.

\subsection{Centeral Divided Difference}

The velocity was estimated with a fixed window of the last 3 position readings.
From there, a centeral divided difference formula was used to estimate the
slope and estimate velocity. This was found to be more accurate than a forward
divided difference method only incorporating two points in practice when run
against simulated data.

The equations that define the centered divided difference are as follows:

\begin{equation}
\dot{\theta} = \dfrac{\theta_{t} - \theta_{t - 2}}{2 \delta t}
\end{equation}

% TODO(buckbaskin): show a plot with the two different estimations side by side to show error

\subsection{Pneumatic Actuator Modeling}

The internal estimation of the current acceleration was primarily based on the 
pressure sensors on the robot. These readings, combined with a position sensor, 
allow for a direct measurements of the state of the actuator and therefore the
applied torque. The model comes from \cite{HuntPMuscles}.

To calculate a pressure from a force and position, the following model is used:

\begin{equation}
P = a_{0} + a_{1} * tan(a_{2} (\dfrac{k}{a_{4} * F + k_{max}} + a_{3})) + a_{5} * F + a_{6} * S
\end{equation}

\begin{equation}
k_{max} = \dfrac{L_{rest} - L_{620 kpa}}{L_{rest}}
\end{equation}

\begin{equation}
k = \dfrac{L_{rest} - L_{angle}}{L_{rest}}
\end{equation}

\begin{equation}
L_{angle} = L_{0} - L_{1} cos(\alpha + \theta)
\end{equation}

\begin{equation}
F = \dfrac{T}{d cos(\beta + \theta)}
\end{equation}

$\alpha, \beta, d, L_{rest}$ and $L_{620 kpa}$ represent the parameters for a 
particular actuator. The subscripted $a$ values represent optimized parameters
for the generalized model that adjusts to all actuators.

There are 3 key aspects to the model. First, the over model is a shifted and 
scaled tangent curve. This represents the observed trend that increasing 
pressure has less strain effects for short changes in length or large changes in 
length, with an stronger effect for medium changes in length. This is similar to 
the behavior of biological muscles.

Second, The scaling effect comes from the observed
effect that increased end effector force leads to a shrinking of the tangent 
curve.

Third, there is a hysteresis term ($a_{6} * S$) that represents the internal
friction observed in the filling and emptying process.

From there, the estimation of acceleration comes from the mass, 
damping and conservative forces and the torque model specified in
\myref{chap:lit_review}.

\section{Optimizing Torque Control}

Once the current state of the robot was determined, the optimization process was 
done recursively using a binary search algorithm. 

This equation and model are designed to present a forward model to convert from 
a desired torque and force to a pressure. There is not a clean and simple 
inversion of the model; however, a binary search algorithm was used to invert 
the process by estimating a maximum and minimum torque to apply and narrowing the
potential window from there. This allows for a estimating a known bound on the 
potential error and balancing the error with how quickly the algorithm iterates.

% TODO(buckbaskin): include the binary search graphic

Bounds were selected from estimates
of the maximum and minimum (maximum negative) torque that can be applied to the
joint and a forward projection mechanism based on the internal physics model was
used to estimate the resulting position. From there, the binary search procedure
repeatedly shrinks the bounds around the optimal torque. The model can be 
applied to each actuator in the pair independently (with a fixed antagonistic 
torque) or as a pair with a variable overlap in torque. 

The final
step is to convert the desired torque into a pressure control. This involves a 
one final forward iteration of the model above, at the current joint angle, to 
calculate the desired pressures for each of the actuators.

\subsection{Forward Projection of State}

Given the current state and an applied torque, the joint model in 
\myref{chap:lit_review} can be used to predict the motion of the joint.
This was done at multiple time resolutions, with increased time resolution 
increasing the accuracy of the internal prediction. This is a direct tradeoff;
however, a more accurate internal prediction directly increases the tracking
performance of the algorithm.

One aspect of the current algorithm that allows for reduced fidelity is the 
opportunity to make estimates for an upper and lower bound. This allows the 
controller to bound the maximum possible positional error and iterate further
if there is insufficient accuracy at particular steps.

\subsection{Torque to Pressure}

Converting from a desired torque to pressure for a single actuator follows
the model expressed above:

\begin{equation}
P = a_{0} + a_{1} * tan(a_{2} (\dfrac{k}{a_{4} * F + k_{max}} + a_{3})) + a_{5} * F + a_{6} * S
\end{equation}

One complicated aspect of implementing conversion from the desired torque to 
pressures for
the joint is incorporating the antagonistic overlap of the actuation of the two 
muscles. It is often sufficient to pick a fixed overlap for the joints; however,
there is room for improvement for varying the overlap and stiffness of the joint.

Another complicated aspect of designing desired pressure in the system is the
knowledge of the underlying control scheme. As mentioned in \cite{HuntPMuscles},
the underlying controller implements a bang-bang algorithm. This means that 
there is some uncontrolled variation; however, a more complicated algorithm
can use the understanding that there is a fixed controlled window

\section{System Modeling}

Parameter weight updates

\subsection{Actual Error Calculation}

% TODO(buckbaskin): adapt existing work on how the error update should be done
This comes from a write up that I've already done

\subsection{Evaluation of asymmetric error concerns}o

% TODO(buckbaskin): use some sort of math to show that the over/underestimation
% of damping, etc. can lead to underdamped or overdamped conditions. Prefer
% overdamped control
