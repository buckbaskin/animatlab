\documentclass[12pt, letterpaper, oneside, onecolumn]{report} % 
% \author{William C. Baskin IV}
% \title{Thesis}
\title{
  {A Biologically Inspired Position Controller for Efficient Trajectory Execution with Antagonistic Pneumatic Muscles}\\
  {\large by \\
  WILLIAM BASKIN \\~\\
  Submitted in partial fulfillment of the requirements\\
  For the degree of Master of Science in Mechanical Engineering\\~\\
  Advisor: Dr. Roger Quinn\\
  Department of Mechanical and Aerospace Engineering\\
  CASE WESTERN RESERVE UNIVERSITY}\\
  % {\large August 2018}
%	{\includegraphics{university.jpg}}
}
\date{August 2018}
% \date{\today}

\usepackage[utf8]{inputenc}
\usepackage[english]{babel}

\usepackage{titling}
\setlength{\droptitle}{-1in}

\usepackage{csquotes}
\usepackage{graphicx}
\graphicspath{ {images/} }
% \includegraphics[height=6.75in,angle=270]{HW25}

\usepackage[lmargin=1in, rmargin=1in, tmargin=1.5in, bmargin=1.5in]{geometry}
\usepackage{setspace}
\doublespacing
\usepackage[style=ieee, backend=biber]{biblatex}
\addbibresource{references.bib}

\usepackage[hidelinks]{hyperref}
\hypersetup{colorlinks=true, anchorcolor=black, linkcolor=black, citecolor=black, urlcolor=blue}
\usepackage{cleveref}

% https://tex.stackexchange.com/questions/119513/cleveref-and-appendix-packages-appendix-referenced-as-section
\crefname{appsec}{Appendix}{Appendices}

\newcommand{\myref}[1]{\hyperref[#1]{\Cref{#1}}}
\newcommand{\github}[1]{\href{https://github.com/buckbaskin/animatlab/tree/master/#1}{#1}}

%%% math %%%
\usepackage{amsmath, amssymb, amsthm}
\DeclareMathOperator*{\argmax}{arg\,max}
\DeclareMathOperator*{\argmin}{arg\,min}

% %%% Code %%%
% \usepackage{color}
% \usepackage{verbatim}
% \usepackage{listings}
% \definecolor{dkgreen}{rgb}{0,0.6,0}
% \definecolor{gray}{rgb}{0.5,0.5,0.5}
% \definecolor{mauve}{rgb}{0.58,0,0.82}

% \lstset{frame=tb,
%   language=Matlab,
%   aboveskip=3mm,
%   belowskip=3mm,
%   showstringspaces=false,
%   columns=flexible,
%   basicstyle={\small\ttfamily},
%   numbers=none,
%   numberstyle=\tiny\color{gray},
%   keywordstyle=\color{blue},
%   commentstyle=\color{dkgreen},
%   stringstyle=\color{mauve},
%   breaklines=true,
%   breakatwhitespace=true,
%   tabsize=3
% }


%%% Packages for the future %%%
% \usepackage{parskip}
% \usepackage{textcomp}
% \usepackage{soul}

\newcommand{\subsubsubsection}{\paragraph}
\newcommand{\bbs}[1]{\section{#1}}
\newcommand{\bbss}[1]{\subsection{#1}}
\newcommand{\bbsss}[1]{\subsubsection{#1}}
\newcommand{\bbssss}[1]{\subsubsubsection{#1}}

\newcommand{\norm}[1]{\left\lVert#1\right\rVert}

\begin{document}
\maketitle

% TODO(buckbaskin): approval pdf

\tableofcontents

\listoffigures

\chapter*{Acknowledgments}
\label{chap:acknowledgements}

I would like to thank the Biologically Inspired Robotics Lab and the faculty of
the EMAE and EECS department for their support and help. I would like to thank my advisor Roger Quinn and Nick Szczecinkski for their guidance and inspiration.

Additionally I'd like to thank my family and friends for their continuing support.

\chapter*{Abstract}
\label{chap:abstract}

This research develops a new synthetic nervous system controller for control of joints using pneumatic actuators in order to create a more efficient and adaptable walking robot system. This design implements new features not previously seen in a control oriented nervous system. The design develops 3 major components not previously used for control of antagonistic pneumatic actuators. The controller uses an internal model of the actuators to estimate the state of the joint. The controller also uses internal estimates of the dynamics of the joint to continually optimize the control output. Additionally, the controller updates its own internal system model with a memory-like loop to allow the controller to adapt to any joint and trajectory. The controller allows for the replacement of proportional or other control designs with a learning system that decreases wasted antagonistic muscle activation and wasted energy, decreases phase lag and increases trajectory tracking accuracy.

\chapter{Introduction}
\label{chap:introduction}

Walking (especially bipedal) is hard

Animals achieve stability and effective walking through redundancy (ex. 4 legs,
6 legs, etc) and multi-level neuron control via sensory feedback and descending
commands. Studies have shown that animals can still exhibit significant portions
of walking behavior without higher level control from the brain and, in certain
conditions, without sensory feedback.

Animals adapt to their environment through physical structure and compliance as
well as adapting through a learning process in the nervous system. While the
simulation or replication of both the physical and neural systems in their
entirety is a very complex task, lessons can be taken from the neural control of
muscles to better control a walking robot.

\section{Task Overview}

This thesis focuses on control of joints for the canine-inspired
Puppy robot in a walking scenario. The two distinguishing physical features of 
the robot are its pneumatic muscle actuators and a design that leans heavily
on the mechanics and skeleton of canines.

% TODO(buckbaskin): insert figure
Figure of Puppy

% TODO(buckbaskin): finish this thought
Pneumatic actuators are used because...

% TODO(buckbaskin): finish this thought
Canine design is used because...

\section{Why Walking?}

Walking robots, and more specifically quadruped walking robots, offer a number 
of advantages over wheeled robots. First, walking robots can cover much more 
varied terrain compared with wheeled vehicles. Second, quadruped and walking
robots with more than 4 legs offer inherent stability compared with biped robots
on the same terrain via redundancy and the ability to move one limb at a time
while keeping the center of mass over a supported area.


On the other hand, walking robots have disadvantages that mean they are not
necessarily the right option for every situation.

Increased energy usage (energy per meter metric?). This comes from the
requirement that the active control in the robot is required to support the
weight of the robot with motors or muscles instead of having weight statically
supported by the structure in the case of wheels.

% TODO(buckbaskin): write about this more
Lower maximum speed


One option that has been explored as an intermediate step are Whegs. These
designs allow for increased climbing ability on uneven terrain while retaining
many of the advantages of wheels including ease of control (although additional
% TODO(buckbaskin): 1-2 citations for Whegs
work must be done to synchronize a tripod gate) and increased speed.


However, canines and other legged animals have adapted to efficient and fast 
methods for walking using muscles that suggest robotic design inspired by 
biomechanics of a dog can offer insights into making a more efficient and 
faster robot that retains the advantages of legged robots for walking on uneven
% TODO(buckbaskin): incorporate a second reason for walking robots
terrain and remaining stable in varied environments.

\section{Specifications and Important Challenges}

For control of a walking robot, based on previous challenges

% TODO(buckbaskin): break this into subsections and a paragraph for each
% challenge
Challenges:
\begin{itemize}
\item Peak position during hip actuation isn't accurate (5-15 deg) 
\item Delays between sensor data off the robot -> lab view -> animatlab -> lab
view -> robot cause tracking problems
\item Scapula drives forward motion for front legs in the reverse Z orientation
\item Improve torque profiles to eliminate external support, even close to
maximum activation
\item Better internals and internal understanding of dynamics to match desired
and intended activations
\end{itemize}

% TODO(buckbaskin): break this into subsections and a paragraph for each
% requirement
Requirements:
\begin{itemize}
\item Increase position accuracy to less than 5 degrees from maximum desired position
\item Minimize phase shift between desired and achieved trajectory
\item Decrease torque errors between neuron desired model and executed torques
\end{itemize}

% TODO(buckbaskin): break this into subsections and a paragraph for each
% design goal
Internal Design Goals:
\begin{itemize}
\item Incorporate and ``understanding" of dynamics
\item Optimize current control for future position or, stated another way,
control for the current time/state with delayed sensory information
\end{itemize}

\section{This Work}

\Cref{chap:controller_design} discusses the design of a new type 
of controller that incorporates the
design goals of understanding the dynamics, projecting forward in time and
optimizing the controlled output based on the understanding of dynamics.

\Cref{chap:neuron_design} discusses the implementation of the controller within 
a neuron
simulation. This discussion talks about some simplifications and approximations
that were made in the process, as well as areas of the network that are more
complex improvements over the controller design in \Cref{chap:controller_design}.



\chapter{Literature Review/Background}
\label{chap:lit_review}
\section{Pneumatic Actuators}

\cite{latexcompanion}

\section{Neuron Design}

\section{Controllers}

\chapter{Controller Design}
\label{chap:controller_design}
The controller design had 3 major goals derived from
requirements and challenges faced with a neural implementation in
\cite{HuntPhDThesis}. See \myref{chap:introduction} for more details.

\begin{enumerate}
\item Increase position accuracy to less than 5 degrees from maximum desired
position
\item Minimize phase shift between desired and achieved trajectory
\item Decrease torque errors between neuron desired model and executed torques
\end{enumerate}

Requirement 2 is fulfilled in
the controller through sensor fusion to estimate state at the time of sensor
readings (assumed to be ``current" time) and then using an internal model of the
joint dynamics to predict the evolution of the joint's position and velocity
given a particular controlling torque. From there, an internal optimization loop
can run to identify a goal torque with sufficient resolution.

An accurate internal model will help meet all 3 requirements. In particular, an
internal model directly solves Requirement 3 by matching controlled torques to
pressures accounting for the pneumatic muscle dynamics. The internal model
designed into the control is designed to adapt to the actual observed
characteristics of the joint, allowing for changing dynamics over time (for
example, adding or removing weight from the robot, degrading pneumatic muscles
or changing positions of other limbs),
adapting to different joints with the same controller and increasing the
robustness of the controller to variations from the original estimated
properties of the robot.

Requirement 1 is met by a combination of the design decisions above and
constitutes a metric by which to measure the effectiveness of the new design
relative to old designs, regardless of implementation.

\section{Sensor Fusion}

Within the controller, state estimation is done in a relatively simplified
manner. This trades off accuracy for computation time, where the estimation was
refined to run sufficiently accurately that the rest of the system was robust to
small estimation errors.

\subsection{Pneumatic Actuator Simulation/Simplification}

The pneumatic actuators and their existing pressure controller were simulated 
at varying degrees of fidelity ...

% TODO(buckbaskin): finish this thought based on code

\subsection{Centered Divided Difference}

The velocity was estimated with a fixed window of the last 3 position readings.
From there, a centered divided difference formula was used to estimate the
slope and estimate velocity. This was found to be more accurate than a forward
divided difference method only incorporating two points in practice when run
against simulated data.

% TODO(buckbaskin): write the two equations

% TODO(buckbaskin): show a plot with the two different estimations side by side
% to show error

\section{Optimizing Torque Control}

Once the current state of the robot was determined, iteratively optimizing the
desired torque was relatively simple. First, bounds were selected from estimates
of the maximum and minimum (maximum negative) torque that can be applied to the
joint and a forward projection mechanism based on the internal physics model was
used to estimate the resulting position. From there, a binary search procedure
can be used to repeatedly shrink the bounds around the optimal torque. The final
step was to convert the desired torque into a pressure control.

\subsection{Forward Projection of State}

Given the state estimation, project forward.

Also, can use bounds on estimated state, say high and low velocity, to project
forward through multiple iterations, and optimize for the torque that, given the
bounds, is most likely to produce the best result. This work is unnecessary for
a well implemented state estimation system, but is helpful for increasing
robustness to potential sensor reading errors or estimation errors.

% TODO(buckbaskin): complete this thought

\subsection{Torque to Pressure}

% TODO(buckbaskin): complete these two steps
Show the math

Also, talk about designing the desired pressure around the knowledge of a
bang-bang controller under the hood for more accurate pressure control

\section{System Modeling}

Parameter weight updates

\subsection{Actual Error Calculation}

% TODO(buckbaskin): adapt existing work on how the error update should be done
This comes from a write up that I've already done

\subsection{Evaluation of asymmetric error concerns}o

% TODO(buckbaskin): use some sort of math to show that the over/underestimation
% of damping, etc. can lead to underdamped or overdamped conditions. Prefer
% overdamped control


\chapter{Neuron Design}
\label{chap:neuron_design}
The synthetic nervous system controller is developed with a <insert name here>
neuron model. The neurons are modeled with a resting potential of -60 mV and a
maximum potential of -40 mV for a range of 20 mV.

% TODO(buckbaskin): make the capitalization of the networks consistent

\bbs{Key Neurons and Synapses}

The neuron controller network is made up of a set of engineered synapses
designed to emulate arthimetic operations. These can be grouped into two
categories: excitatory neurons and inhibitory neurons. Most neurons were tuned
via an optimization process to identify the best equilibrium potential and
synaptic conductance. Some neurons were tuned by hand for specific behaviors in
a subsection of the overall controller.

\bbss{Excitatory Synapses}

Except where otherwise noted, a value of 134 mV was used for the equilibrium
potential. The pre-synaptic threshold is -60 mV and the pre-synaptic saturation
level is -40 mV.

\bbsss{Signal Transfer}

The signal transfer synapse is designed to pass the voltage of the input neuron
to the output neuron in the active range of the neuron. It is often used to add
the value of two or more neurons together in an output neuron.
It has a synaptic 
conductance of 0.115 microsiemens. 

\bbsss{Inverted Signal Transfer}

By combining two signal inversions, a more accurate signal transfer synapse was
created. This involves an extra neuron; however, it leads to a more precise
transfer of the voltage level of the input neuron to the output neuron. In
general, the synapse was implemented with two Signal Inverter (Stimulated)
synapses.

\bbsss{Signal Amplifier 2x}

The signal amplifier synapse is designed to pass the voltage of the input neuron
to the output neuron with a 2x gain. This was tuned with input values from 0 to
10 mV. Higher inputs saturate the output neuron. It has a synaptic conductance
of 0.23 microsiemens.

\bbsss{Signal Amplifier 4x}

The signal amplifier has the same function as the 2x amplifier but with a x4 
gain. This was tuned with input values from 0 to
5 mV. Higher inputs saturate the output neuron. It has a synaptic conductance
of 0.46 microsiemens.

\bbsss{Signal Reduction 0.2x}

The signal reduction synapse is designed to pass the voltage of the input neuron
to the output neuron in the active range of the neuron, but with a loss of 0.2x. 
It has a synaptic  conductance of 0.021 microsiemens. This neuron was tuned
over the complete range of input values, 0 to 20 mV.

\bbsss{Signal Reduction 0.5x}

The signal reduction synapse is designed to pass the voltage of the input neuron
to the output neuron in the active range of the neuron, but with a loss of 0.5x. 
It has a synaptic  conductance of 0.054 microsiemens. This neuron was tuned
over the complete range of input values, 0 to 20 mV.

\bbsss{Convert Forward Positive}

The convert forward positive synapse is designed to convert the representation
of a value in a single neuron, for example the neuron that represents the 
position of the joint, to the same value represented in two neurons. This
synapse converts the value when the value is above -50 mV to values between 0 
and 20 mV. The pre-synaptic
threshold is -50 mV and the synaptic conductance is 0.115 microsiemens.

\bbsss{Torque Pressure Converter}

The torque pressure converter synapse is designed to perform the torque-pressure
calculation approximation. It has a synaptic 
conductance of 0.048 microsiemens.

\bbss{Inhibitory Synapses}

Except where otherwise noted, the equilibrium potential of inhibitory neurons
is simulated as -100 mV. This is the lowest value possible in the simulation.
The pre-synaptic threshold is -60 mV. The pre-synaptic saturation level is -40
mV.

\bbsss{Signal Inverter}

The signal inverted synapse is designed to decrease the voltage of the output 
neuron proportional to the voltage of the input neuron. It has a synaptic 
conductance of 0.55 microsiemens.

\bbsss{Signal Inverter (Stimulated)}

The signal inverted synapse is designed to decrease the voltage of the output 
neuron proportional to the voltage of the input neuron. This synapse is tuned
slightly differently from the standard signal inverter because there was an
observed difference in voltage when a stimulus current was applied to the output
neuron along with other incoming synapses. It has a synaptic 
conductance of 0.5 microsiemens.

\bbsss{Signal Invert Reduction 0.2x}

The signal inverted reduction synapse is designed to decrease the voltage of
the output 
neuron at a 0.2x loss compared to the voltage of the input neuron. It has a 
synaptic conductance of 0.093 microsiemens.

\bbsss{Signal Invert Reduction 0.5x}

The signal inverted reduction synapse is designed to decrease the voltage of
the output 
neuron at a 0.5x loss compared to the voltage of the input neuron. It has a 
synaptic conductance of 0.22 microsiemens.

\bbsss{Signal Inverter Amplifier 2x}

The signal inverted amplifier synapse is designed to decrease the voltage of
the output 
neuron at a 2x gain to the voltage of the input neuron. It has a synaptic 
conductance of 1.11 microsiemens.

\bbsss{Signal Inverter Amplifier 4x}

The signal inverted amplifier synapse is designed to decrease the voltage of
the output 
neuron at a 4x gain to the voltage of the input neuron. It has a synaptic 
conductance of 2.3 microsiemens.

\bbsss{Integral Inhibitor}

The integral inhibitor synapse is designed to help both neurons maintain a 
stable value unless an external current is applied. The values for this
synapse are based on \cite{NickFunctionalSubnetwork}. The synapse has a  
conductance of 0.5 microsiemens.

\bbsss{Convert Forward Negative}

The convert forward negative synapse is designed to convert the representation
of a value in a single neuron, for example the neuron that represents the 
position of the joint, to the same value represented in two neurons. This
synapse converts the value when the value is below -50 mV to a positve value 
from 0 to 20 mV. The pre-synaptic
saturation is -50 mV and the synaptic conductance is 0.5 microsiemens.

\bbsss{Signal Divider}

The signal divider synapse is designed to reduce the effect of another input
synapse, where the reduction increases with increased input voltage to the
signal divider synapse. It is distinguished from the signal multiplier synapse
in that the value never reaches 0. Intuitively, this is a replication of 
division where dividing by a large number makes the quantity small but never 0.
It has a synaptic conductance of 20 microsiemens. The equilibrium potential of
the synapse is -60 mV (equal to the resting potential of the input and output
neurons).

\bbsss{Signal Multiplier}

The signal divider synapse is designed to reduce the effect of another input
synapse, where the reduction deccreases with increased input voltage to the
signal divider synapse. It is distinguished from the signal divider synapse
behavior in that the value reaches 0. Intuitively, this is a replication of 
multiplication where multiplying by 0 will make any value 0.
It has a synaptic conductance of 19.75 microsiemens. The equilibrium potential 
of the synapse is -61 mV.

\bbs{Sensor Fusion}

The Sensor Fusion neuron network performs essentially the same function as the
sensor fusion network in the prototype controller. In this case, 3 neurons
represent the 3 sensor inputs available in a joint: position (``Theta"),
extension muscle pressure (``Ext Pres") and flexion muscle pressure
(``Flx Pres"). The outputs for the network are the estimates for current 
position, velocity and acceleration.

\bbss{Velocity Fusion Network Components}

\bbsss{Differentiator Network}

% 2
% TODO(buckbaskin): talk about differentiator network
The velocity network is based on the Differentiator Network presented in 
\cite{NickFunctionalSubnetwork}.

% 3
% TODO(buckbaskin): Figure of differentiator network

\bbss{Velocity Fusion Network}

The velocity network is based on the Differentiator network. 
The one major change from the network,
as presented, is the inclusion of a second $U_{post}$ neuron to represent the
negative derivative of the position (negative velocity).

% 4
% TODO(buckbaskin): Figure of my velocity network

% TODO(buckbaskin): make sure my inhibitory and excitatory synapses are visualized properly

This represents
a common pattern used across the network where two neurons are used to represent
a single value. One neuron represents positive levels of the variable and is 
at or below resting potential when the value is negative. The other neuron is
above resting potential for negative values and is at or below resting potential
when the value is positive. 

The motivation for increasing the complexity of the
network (often making it more than twice as complicated) is to increase the
effective range of values that the neuron can represent at the same fidelity and
to increase the accuracy of zero. When a single neuron represents positive and
negative values of equal magnitude, the value of 0 is represented at 50 mV; 
however, after passing through a signal transfer synapse this value is often
slightly higher, up to 52 mV. This means that comparing the two neurons (the
original neuron and the signal transfer) yields a slightly positive error
instead of near zero error.

\bbss{Acceleration Fusion Network Components}

\bbsss{Absolute Value Network}

Within the acceleration fusion network, the absolute value of the position is
used. This is calculated by first splitting the position into its two neuron
representation. From there, the sum of the two neurons is used as the absolute
value. This takes advantage of the definition of each side of the two neuron
representation falling below resting potential when the other neuron is active.
This means the signal transfer synapse from the below zero neuron will have no
effect.

% 5
% TODO(buckbaskin): Figure of my absolute value network

\bbsss{Integration Network}

The integration network used throughout the neuron controller is based heavily
on the Integrator Network in \cite{NickFunctionalSubnetwork}. Two neurons are
designed to mutually inhibit each other so that the combined pair hold their
values. Individual neurons can be treated as a leaky integrator; however, their
voltage tails off over time if there is no maintenance current. The integration
network itself is tuned by changing the time constant of the component neurons
to adjust how much the voltage of the integrator network changes for an input
current.

\bbsss{Convert Torque to Pressure}

% 6
% TODO(buckbaskin): talk about how this approximation came about through simulation
Linearized approximation to help with neurons

% TODO(buckbaskin): regenerate the PressureTorque figure with axis labels
\begin{figure}[h!]
\centering
\includegraphics[width=5in]{neuron_design/FigPressureTorque}
\caption{Torque/Pressure relation observed in simulation}
\label{fig:PressureTorque}
\end{figure}

\bbsss{Pressure Estimation Loop}

Within the Acceleration Network, there is a feedback loop that is used to 
estimate the torque applied by a given pressure. First, the loop ``initializes"
with an extension torque guess from the integrator. Second, the extension
torque is converted into extension pressure. Third, the estimated extension
pressure is compared with the sensed pressure. If there is an delta between the
two, the extension torque guess is modified in turn and the cycle repeats.

% 7
% TODO(buckbaskin): Figure of pressure loops

This architecture is mirrored for flexion torque.

\bbss{Torque to Acceleration Network}

% 8
% TODO(buckbaskin): talk about T2A

\bbss{Acceleration Fusion Network}

The acceleration fusion network is a combination of an integrator, absolute
value network, the pressure estimation loop and a torque to acceleration
network. In total, the network uses a combination of smaller networks to
estimate the torque applied from sensed pressure and then combine the extension
and flexion torques together to get a net torque and acceleration.

\section{Optimizing Torque Control}

% 9
I/O summarizing, goals
position, velocity, desired position -> torque

\subsection{Small Networks}

% 10

\subsection{Mid Networks}

% 11

\subsection{Entire Network}

% 12

\section{Torque to Pressure}

% 13
I/O summarizing, goals
convert desired torque pressure in neurons

\subsection{Small Networks}

% 14

\subsection{Mid Networks}

% 15

\subsection{Entire Network}

% 16

\section{System Modeling}

% 17
I/O summarizing, goals
Parameter weight updates

\subsection{Small Networks}

% 18

\subsection{Mid Networks}

% 19

\subsection{Entire Network}

% 20

\bbs{Combined Controller Network}

\begin{figure}[h!]
\centering
\includegraphics[width=5in]{neuron_design/NetworkLayout}
\caption{Overview of major network components}
\label{fig:NetworkLayout}
\end{figure}
% TODO(buckbaskin): redo this with the complete actual network

\chapter{Testing/Methods}
\label{chap:methods}
\section{Simulation}

\subsection{Animatlab}

Animatlab is a simulation environment that combines a physics engine with a
neuron simulation tool. It is very helpful for integrating physical systems
and sensor information easily into a synthetic nervous system.

The design tools for defining and testing a nervous system were also instrumental
to the success of the new controller design.

\subsection{Python Numerical Simulation}

During the development of the original controller design, an additional
simulation environment was created in Python. Due to the complicated nature of
the controller design, it is impractical to prove the stability and other
properties of the controller analytically, especially with changing internal
parameters. Instead, the Python simulation environment was used to gradually add
in expected physical effects to see how the algorithm behaved and demonstrate
stability in practice.

The pneumatic actuators and their existing pressure controller were simulated 
at varying degrees of fidelity. Originally, a simple model of the actuators
was used that emulated their non-linear behavior for applying torque, but didn't
include damping effects, load or limitations of the underlying pressure
controller. Over time these effects were added in until a high fidelity model of
a joint was controlled with the algorithm. This simulation environment also
enabled the testing and verification of a simplified internal model for
replication within the neuron controller. Without the Python simulation
environment, this evidence-based design of models would not have been as easy to
iterate.

\section{Testing and Verification}

One of the key steps for verifying the neuron controller was a series of tests
using the DataTool functionality of Animatlab. By simulating inputs to subsets
of the network, the output can be verified against the Python controller
implementation or other pre-computed values. This independent testing further
reinforces the value of the functional sub-network approach.

\subsection{Neurons}

One of the most important aspects of the functional subnetwork approach is that
individual subnetworks can be tuned, verified and then combined with the larger
network. In order to better tune individual networks, a testing rig was set up
within the larger network to allow for subnetwork verification. The typical
interface neurons are disabled and special test driver neurons (highlighted in
yellow) are enabled. This allows for the driving of custom input signals and
combinations of signals in order to verify the output value and speed. 
See \myref{fig:TestNetworkT2P} for a good example.

\begin{figure}
\centering
\includegraphics[height=3.5in]{methods/TestVisualT2P}
\caption{Torque to pressure network with test and actual input neurons shown}
\label{fig:TestNetworkT2P}
\end{figure}

This particular network is designed to convert a desired torque from the rest
of the network to the pressure for the extension and flexion pneumatic actuators.
With this network, the input neurons (in purple)
are disabled for testing and the test neurons are enabled. From there, inputs
are driven, as seen in \myref{fig:TestNetworkInputs}. This allows the inputs to 
be mapped to real world values, the real world inputs to be mapped to outputs, 
and the real world outputs to be mapped back to neuron voltages. Through this
a test framework is built where the output can be visually compared to the 
reference to ensure neuron networks are working as intended. See 
\myref{chap:results} for the results of the tests that were run.

\begin{figure}
\centering
\includegraphics[height=3.5in]{methods/T2PInput}
\caption{Drive inputs to test the pressure network}
\label{fig:TestNetworkInputs}
\end{figure}

The test frameworks can be rigged at multiple levels to enable testing of
individual units and the integration of units in the style of unit tests and
integration tests from a more traditional software background.


\chapter{Results}
\label{chap:results}
\bbs{Simulation}

Antagonistic actuators were simulated at various pressures to better understand
their behavior using \github{stability/constant\_pressure.py}. In \myref{fig:AntagonisticPressureTorque}, the extension
actuator was set to a constant 500 kPa. The pressure in the flexion actuator was
varied, and the torque was plotted for a range of actuator angles. This model of
torque generated was used in ongoing work to help better understand joint
dynamics and improve the performance of the controllers.

\begin{figure}
\centering
\includegraphics[height=3.5in]{results/Pos_v_AntagTorque}
\caption{Relation between joint position, actuator pressures and
torque applied. Extension actuator pressure was set to 500 kPa and flexion was
set as indicated.}
\label{fig:AntagonisticPressureTorque}
\end{figure}

During the characterization of the joint, simulations were run to determine the
dominant effects between inertia, damping and static loads using \github{stability/max\_torque.py}. The relative 
magnitudes of the acceleration, velocity and position are shown in 
\myref{fig:MaxTorque}, along with the
torque. For the simulated joint, the dominant effect was the inertia; however,
for a robot carrying varying loads, the inertia will not always be the dominant factor. The 
maximum torque required to execute the known trajectory is also later used as a
metric for measuring the efficiency of the controller. A controller that is too
aggressive for the trajectory (ex. too high gains on a proportional controller)
will request larger than needed torques during trajectory execution. An ideal
controller will always follow the trajectory with the exact torque needed during steady state operation.

\begin{figure}
\centering
\includegraphics[height=3.5in]{results/Max_Torque}
\caption{Comparing the effects of inertia, damping and static loads. Torque follows acceleration, which suggests that inertia dominates for this simulation}
\label{fig:MaxTorque}
\end{figure}

\bbs{Python Controller}

The prototype controller written in Python was used to demonstrate the stability
and tracking capabilities of the general algorithm. The prototype controller also was used to find a
suitable simplification of the internal model that was sufficiently accurate for
optimization but had minimal parameters to tune.

\bbss{Simplified Controller}

The simplified controller demonstrated effective control once the weight updates
were combined with a suitable starting state. \myref{fig:SimplifiedTracking}
demonstrates the progression from original state to a stable and accurate
control of joint position. The blue line is the desired trajectory, with purple
used to highlight error bounds of $\pm$ 1 degree. Orange indicates the actual
position of the joint, and the green line indicates the internal state
estimation.

\begin{figure}
\centering
\includegraphics[height=4in]{results/State_Estimation}
\caption{Tracking improves with improved state estimation and internal model
updates}
\label{fig:SimplifiedTracking}
\end{figure}

\bbss{Static Controller}

During testing, the need for an internal model update was verified through 
tests where the optimization loop was applied with variations on 
parameters using \github{stability/simple\_mass\_model.py}. One example was varying mass. For correct values, such as 
\myref{fig:StateEstimationPerfect}, the system worked within tolerances. When 
mass was underestimated (see \myref{fig:StateEstimationLowMass}), the tracking 
performance degrades to smoothly fail required tracking accuracy. On the other 
hand, over-estimating the inertia caused aggressive oscillation and underdamping (see \myref{fig:StateEstimationHighMass}). Based on these tests, the assumption 
that the internal model needs to update was validated.

\begin{figure}
\centering
\includegraphics[height=4in]{results/State_Estimation_Perfect}
\caption{Accurate tracking with a good internal estimation of mass}
\label{fig:StateEstimationPerfect}
\end{figure}

\begin{figure}
\centering
\includegraphics[height=4in]{results/State_Estimation_LowMass}
\caption{Overdamped tracking with an internal underestimation of mass}
\label{fig:StateEstimationLowMass}
\end{figure}

\begin{figure}
\centering
\includegraphics[height=4in]{results/State_Estimation_HighMass}
\caption{Underdamped failure with an internal overestimation of mass}
\label{fig:StateEstimationHighMass}
\end{figure}

\bbss{Simple Parameter Estimation}

The original model for parameter updates was to calculate the sign of the update
and make a uniform incremental change in that direction. In practice, the constant uniform update
resulted in near-perfect tracking; however, the weight updates themselves were continually varying instead of stabilizing to a true value. See \myref{fig:StateUpdateSimple} and \github{stability/simple\_mass\_model.py}.

\begin{figure}
\centering
\includegraphics[height=7.5in]{results/State_Model_SimpleUpdate}
\caption{Successful tracking with simple parameter update}
\label{fig:StateUpdateSimple}
\end{figure}

\bbs{Neuron Controller}

Animatlab does not offer a good model of pneumatic muscles for use in closed loop testing. In place of this testing, the Python simulation was used to simulate the behavior of the pneumatic actuators and the neuron model was tuned to try and match the output of the Python controller from the small component subnetworks up to the overall controller behavior. The standard for a good match was less than 10 percent error across the simulated range of neuron inputs. For neurons with an active range between -60 mV and -40 mV, the determination for a good network is an average error of less than 2 mV. 

\bbss{Test Results}

Test results are generated based on data tools in \github{PuppyNeuronPlayground}. Post processing and visualization is done with scripts found in \github{writeup/scripts}.

\begin{figure}
\centering
\includegraphics[height=2.25in]{results/TestVelPosWide}
\caption{Testing positive velocity estimation}
\label{fig:TestVelPos}
\end{figure}

\begin{figure}
\centering
\includegraphics[height=2.25in]{results/TestVelNegWide}
\caption{Testing negative velocity estimation}
\label{fig:TestVelNeg}
\end{figure}

\begin{figure}
\centering
\includegraphics[height=3.5in]{results/New_T2A}
\caption{Testing conversion from torque to acceleration with varying link inertia. Mean error is 0.6 mV}
\label{fig:TestAccelInertia}
\end{figure}

\begin{figure}
\centering
\includegraphics[height=3.5in]{results/New_T2A_2} % TODO(buckbaskin): complete this thought with a picture and data
\caption{Testing conversion from torque to acceleration, with varying joint damping. Mean error is 0.7 mV}
\label{fig:TestAccelDamping}
\end{figure}

\begin{figure}
\centering
\includegraphics[height=3.5in]{results/New_P2T} % TODO(buckbaskin): complete this thought with a picture and data
\caption{Testing conversion from actuator pressure to torque. Mean error is 1.4 mV}
\label{fig:TestP2T}
\end{figure}

% TODO(buckbaskin): does it make sense to add a damping graphic in here?

\begin{figure}
\centering
\includegraphics[height=3.5in]{results/New_TO}
\caption{Estimating positive torque required to reach a desired position. Mean error is 0.3 mV}
\label{fig:TestTorqueOptimizationPos}
\end{figure}

\begin{figure}
\centering
\includegraphics[height=3.5in]{results/New_TO_2}
\caption{Testing negative torque required to reach a desired position. Mean error is 2.7 mV}
\label{fig:TestTorqueOptimizationNeg}
\end{figure}

\begin{figure}
\centering
\includegraphics[height=3.5in]{results/New_T2P}
\caption{Estimated extension pressure from desired torque. Mean error is 1.6 mV}
\label{fig:TestT2PPos}
\end{figure}

\begin{figure}
\centering
\includegraphics[height=2.25in]{results/TestSystemCPosWide}
\caption{Positive neuron and actual damping factor updates}
\label{fig:TestSystemCPos}
\end{figure}

\begin{figure}
\centering
\includegraphics[height=2.25in]{results/TestSystemCNegWide}
\caption{Negative neuron and actual damping factor updates}
\label{fig:TestSystemCNeg}
\end{figure}

\begin{figure}
\centering
\includegraphics[height=2.25in]{results/TestSystemNPosWide}
\caption{Positive neuron and actual load factor updates}
\label{fig:TestSystemNPos}
\end{figure}

\begin{figure}
\centering
\includegraphics[height=2.25in]{results/TestSystemNNegWide}
\caption{Negative neuron and actual load factor updates}
\label{fig:TestSystemNNeg}
\end{figure}

\bbsss{Sensor Fusion}

Two key estimates are made within the sensor fusion subnetwork: velocity and
acceleration. The synthetic nervous system controller was tested against sine waves of known
frequencies. The positive and negative estimated are compared to the ground truth in \myref{fig:TestVelPos} and \myref{fig:TestVelNeg}.
Within the neuron network, acceleration is calculated primarily from actuator
pressures. For testing, this network was split into two components: estimating
torque based on actuator pressure (\myref{fig:TestP2T}) and estimating acceleration from torque. The
output of the neuron model is compared to the expected output in
\myref{fig:TestAccelInertia} and \myref{fig:TestAccelDamping}. All 3 tests met the standard, with
mean errors of 1.4 mV, 0.6 mV and 0.7 mV.

% TODO(buckbaskin): check figures in here

\bbsss{Torque Optimization}

The torque optimization network was tested as a complete network. Taking the entire network as a whole, the inputs (position, velocity and desired position) are driven to input values and the outputs are compared with the torque predictions of the prototype controller. This is a higher level test designed to comprehensively test the connectivity between smaller subnetworks examined in other tests. The results of the tests are shown in \myref{fig:TestTorqueOptimizationPos} and \myref{fig:TestTorqueOptimizationNeg}. The positive network is quite accurate with a mean error of 0.3 mV; however,
the error for the negative torque prediction was higher, at 2.7 mV. Due to the difference between positive and negative results, this suggests that there may be significant enough difference between signal transfer and inverter synapses such that, over the course of a longer loop, the accuracy diverges.

The components for converting the torque to
acceleration also were tested
separately. See \myref{fig:TestAccelInertia} and \myref{fig:TestAccelDamping}. As discussed in the Sensor Fusion results, the acceleration calculations were accurate to within 1 mV on average across the range of data points.

The components for converting the torque to pressure also were tested
separately. See \myref{fig:TestT2PPos}.
The test results indicated an average error of about 1.6 mV across the entire
working range of the joint.

\bbss{System Modeling}

The system modeling network was tested as a complete network because both the
damping and load factors are updated from the same $\lambda$ value. See 
\myref{fig:TestSystemCPos}, \myref{fig:TestSystemCNeg},
\myref{fig:TestSystemNPos} and \myref{fig:TestSystemNNeg}.

\bbss{Discussion}
\label{chap:discussion}

% TODO(buckbaskin): edit discussion to reflect "new" data

In this thesis, we propose a new system for controlling joints actuated with
pneumatic artificial muscles. The design focuses on improving controller
performance through improved sensor processing, an internal optimization step
and an observer that continually updates the internal physics model to improve
the optimization step. This design was implemented as part of a synthetic
nervous system controller, with modifications made to the initial design to aid
in conversion to a neuron and synapse model and to take advantage of the benefits of the
synthetic nervous system approach.

\bbsss{Sensor Fusion}
\bbssss{Velocity}

The velocity network in the sensor fusion algorithm varies the most from the
reference data (see \myref{fig:TestVelPos} and \myref{fig:TestVelNeg}). This is a known phenomena addressed in
\cite{NickFunctionalSubnetwork}. This variance is observed in two quantities,
the phase shift and the magnitude.

In the velocity network, the magnitude of the positive and negative
velocities varied quite widely compared with the reference, especially at
higher gains. During the tuning of the network, a balance was chosen between
the maximum phase lag, minimized by decreasing the time constant of the slower
neuron, and increasing the synaptic gain (and therefore the observed difference between
positive and negative velocities).

This error suggests that there is an underlying error in the way that positive
and negative operations are computed; however, through individual tuning the
synapses appear to be relatively accurate. The use of the derivative network
varies in each application across different subnetworks. The engineering
solution for the correct balance of accuracy, gain and phase varies. This
variance leaves open the opportunity for biology to inform a single true correct
solution or a more complex derivative network that can independently select for
gain and phase.

\bbssss{Acceleration}

The acceleration network converges to a near-correct solution.
This convergence demonstrates that the feedback loop concept can be effective; however, the process takes a relatively long, fixed amount of time to
converge from solutions that are a significant step away. This time delay may lead to
issues with rapidly changing pressures on a hardware system. One potential
solution is to modify the time constants of the network
elements within the loop, in particular the integration network, to allow the
internal loop(s) to run faster than the rest of the network
so the outputs of the internal loop appear to converge to the correct value on the same time scale as the
rest of the network. 

This feedback loop design pattern behavior matches the behavior of a synthetic nervous system implementation of adapting force walking in \cite{LegLocal}. This feedback pattern is effective for local learning of control adaptations in synthetic nervous systems and also represents an adaptation of feedback behavior observed in insects \cite{LegLocal}.

\bbsss{Torque Optimization}

In many cases the torque optimization network converges to the
correct solution; however, like the feedback in the acceleration network, the torque optimization process can
take a relatively long time to converge to the correct solution from discontinuous starting
values. Due to the integrator included in the torque optimization network, the starting conditions
for the first test case are based on other tests and did not show consistent
behavior.

The concept of the torque optimization network appears to be effective; however,
its implementation in neurons suffers from both attempting to converge to a
solution quickly and approximating a physical simulation internally. The torque optimization network
uses relatively low time constants and high gains. This can cause overshoot
(seen in \myref{fig:TestTorqueOptimizationPos}) where the
estimated torque is off the charts for either large positive or negative swings.
The high gains also reduce accuracy by amplifying small errors made during the
approximation of the ``physics" implementation. This behavior replicates
observations made during the development of the prototype controller in code,
where an attempt to approximate the physics of the joint in a single step was
inaccurate. The solution to the simulation error was to make multiple iterations by subdividing
the simulated time so that a simpler physics model would have a closer fit to
the actual dynamics. The iteration in the prototype controller suggests that a longer string of neurons would be
necessary within the loop to project dynamics forward; however, increasing the length of the feedback loop may accentuate existing delays when interfaced to hardware.

\bbssss{Acceleration from Torque}

The 3 stage network for estimating the acceleration from the torque applied to
the system appears to work well. There are two test cases, between 7 and 9
seconds that were not significant when tested at the resolution that the
network incorporates (shown in \myref{fig:TestT2APos} and \myref{fig:TestT2ANeg}). Otherwise, the acceleration is almost always correct.
Given its intended uses, the network is sufficiently accurate and suggests that
relatively small sets of neurons can provide a representation of the physics of
a biological system. This would suggest that an animal's nervous system can understand the dynamics of its environment and its body.

\bbssss{Estimating Pressures from Torques}

The process for calculating pressures from torques is accurate in some
cases and inaccurate in many others. This varying accuracy suggests that the physical
model of the pneumatic artificial muscles proposed in \cite{HuntPMuscles}
does not convert well to a relatively small network of neurons. A future controller iteration may elect to move the
calculation of this conversion out of neurons and
to LabView other other code used to interface between synthetic neurons and
hardware.

\bbsss{System Model}

The network for predicting the weight update is very accurate and suggests that
the engineered process of estimating the modeling errors of a synthetic neuron
network within the network itself is both feasible and an effective process for
the continual improvement of a controller applied to a particular hardware
joint under control.

This synthetic nervous system application also is generally more suitable to neuron networks than say,
the calculation of the conversion from pressures to torques following a well
prescribed algorithm involving a shifted tangent function. The exact magnitude
of the update of the weights, especially at a reduced gain, is not as important
as the relative magnitude and sign of each update. This characteristic means that the
implementation of an arithmetic operation or dynamic calculation by a synapse of
a collection of synapses is tolerant to small variances from the expected
behavior.


\chapter{Conclusions and Future Work}
\label{chap:conclusion}
\bbs{Conclusion}

This thesis presents the design of a new style of biologically inspired
controller for controlling revolute joints with antagonistic pneumatic
artificial muscles. First, the literature surrounding the characterization of
the actuators and the methods used to control them is discussed in
\myref{chap:lit_review}. Second, the
design of a prototype for a controller is discussed in
\myref{chap:controller_design}. Third, the implementation of the design in a
synthetic nervous system is discussed in \myref{chap:neuron_design}. The methods
for testing the controller are discussed in \myref{chap:methods}. The results of
the tests are presented in \myref{chap:results}.

The controller implements new features not previously used in synthetic nervous
system design and not previously used for control of joints actuated by
pneumatic artificial muscles. The system has a whole has been shown to offer
increased accuracy in joint position and decreased phase shift. This was
combined with a better internal model of the actuators themselves to more
accurately model the forces and torques applied in order to continue successful
operation even near the maximum output of the actuators. On the other hand, the
successful operation of the controller is dependent on the tuning of a number of
internal parameters that are sensitive to slight overestimates that lead to
oscillation and loss of control of the joint. This has been mitigated by
asymmetric model updates that favor stable error and starting from known stable
estimates; however, this leads to tracking that does not meet the accuracy goals
discussed in the design when the parameters are too conservative.

The implementation of the controller in a synthetic nervous system led to the
design of unique nervous system functional subnetworks; however, some networks
were found to be inconsistent with their standard controller output, leading to
potential deficiencies in observed performance. This would suggest that a more
bionic approach that leverages mathematical models directly for complicated
calculations, such as the relation between pressure and torque for pneumatic
artificial muscles, and the dynamic benefits of a synthetic nervous system for
aspects of the controller that require less precision and benefit from insights
gained from a deeper biological understanding of how animal nervous systems
control joints.

\bbs{Future Work}

The next step for the design of the controller is experimental testing to
characterize performance on actual hardware. During simulation and testing,
many aspects of the controller worked well; on the other hand, some did not
behave as well as expected. On hardware, there is always some variation that may
show that the design choices made were effective and practical solutions.

There is also some potential for improvement of the design of the controller.
Further analysis of the stability of the controller may lead to a better model
update procedure. In particular, algorithms such as an Extended Kalman Filter
are used in many applications across robotics for estimating parameters from
sensor data, but the Kalman Filter algorithm itself was not used because it does
not have an effective way to be implemented within a synthetic nervous system.
New biological research % TODO(buckbaskin): cite that paper with the toroid structure
suggests that animals have groups of neurons that estimate orientation and
% TODO(buckbaskin): and that time and space cells paper thing I found
position in time and space. This research suggests potential approximations for
spatial estimation that may be useful for estimation of other parameters to
emulate or replace the use of a Kalman Filter.

Another area where the neuron controller has room for improvement is
implementation of certain subnetworks that don't effectively implement their
counterparts in the prototype. There were approximations made to match neuron
and synapse behavior at a low level to some of the mathematics; however, taking
a larger system approach with an aim to design neuron connections to emulate
behavior of larger pieces of the system made lead to a more successful design.
The end goal is to design a working controller with similar properties, this
doesn't need to be achieved by direct copying. The testing methods implemented
in this thesis offer a good system for iterating on this kind of design. Test
inputs can be set up once and then higher level component behavior can be
compared quickly and visually.

Overall, the design of the controller presents a new style of design that can be
used to create optimal controllers for actuation systems that are highly
non-linear and tend to perform poorly when simpler controller designs are used.
There is room for improvement, but the methods used in the thesis offer a
framework for improvement and determining the areas with the largest impact.


\newpage
\appendix
\crefalias{section}{appsec}
\chapter{Code and Resources}
\label{app:resources}
The code and configuration files for this project are available on Github at \url{https://github.com/buckbaskin/animatlab}. In general, each section corresponds to a directory in the repository (the title links to the folder).

\bbs{\github{PuppyNeuronPlayground}}

This folder contains the Animatlab project used to test and design the synthetic nervous system controller. It includes a number of data tools for testing and development.

\bbs{\github{stability}}

This folder contains most of the Python simulation resources used for the project. It was originally named for its intention to measure and visualize the stability of controllers, but it also contains other work as well. Key files are \texttt{constant\_pressure.py}, \texttt{max\_torque.py}, \texttt{pressure\_torque.py}, \texttt{reduced\_controller.py}, \texttt{simple\_mass\_model.py}, \texttt{simulation.py}, and \texttt{torque\_projection.py}.

\bbs{\github{fitting\_neurons}}

This folder contains Python scripts used to fit the neuron model in \cite{NickFunctionalSubnetwork} to arithmetic operations.

\bbs{\github{testing}}

This folder was used to plan out testing setups for use in the controller tests.

\bbs{\github{writeup}}

This folder contains the materials used to create the full thesis writeup as well as a conference paper.

\bbss{\github{writeup/data}}

This subfolder contains the raw data exports from Animatlab as well as the Python script to convert the data to a more common \texttt{.csv} format and the converted CSV files.

\bbss{\github{writeup/scripts}}

This subfolder contains scripts used to more consistently visualize Animatlab data using \texttt{matplotlib}, a Python library.

\bbs{\github{Poster}}

This folder contains the materials used to create a poster for Case Western's Research Showcase.

\bbs{\github{ICanDoMath}}

This was a small Animatlab project used to verify the tuning of synapses in Python. Some of this functionality was ported to the main design project.

\bbs{\github{docs}}

This folder contains notes from eariler in the project.

\newpage
\label{chap:references}
%TODO(buckbaskin): clean out extra unnecessary information from references.bib
\printbibliography[heading=bibintoc, title={Bibliography}]

\end{document}
