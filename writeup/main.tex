\documentclass[12pt, letterpaper, oneside, onecolumn]{report} % 
\author{William Baskin}
\title{Thesis}
\title{
  {A Biologically Inspired Controller for Efficient Trajectory Execution with Pneumatic Actuated Joints}\\
  {\large Case Western Reserve University \\
  Department of Mechanical and Aerospace Engineering}%\\
%	{\includegraphics{university.jpg}}
}
% \date{Day Month Year}
\date{\today}

\usepackage[utf8]{inputenc}
\usepackage[english]{babel}
\usepackage{graphicx}
\graphicspath{ {images/} }
% \includegraphics[height=6.75in,angle=270]{HW25}

\usepackage[margin=1in]{geometry}
\usepackage{setspace}
\doublespacing
\usepackage[style=ieee, backend=biber]{biblatex}
% \addbibresource{references.bib}

% %%% math %%%
% \usepackage{amsmath, amssymb, amsthm}
% \DeclareMathOperator*{\argmax}{arg\,max}
% \DeclareMathOperator*{\argmin}{arg\,min}

% %%% Code %%%
% \usepackage{color}
% \usepackage{verbatim}
% \usepackage{listings}
% \definecolor{dkgreen}{rgb}{0,0.6,0}
% \definecolor{gray}{rgb}{0.5,0.5,0.5}
% \definecolor{mauve}{rgb}{0.58,0,0.82}

% \lstset{frame=tb,
%   language=Matlab,
%   aboveskip=3mm,
%   belowskip=3mm,
%   showstringspaces=false,
%   columns=flexible,
%   basicstyle={\small\ttfamily},
%   numbers=none,
%   numberstyle=\tiny\color{gray},
%   keywordstyle=\color{blue},
%   commentstyle=\color{dkgreen},
%   stringstyle=\color{mauve},
%   breaklines=true,
%   breakatwhitespace=true,
%   tabsize=3
% }


%%% Packages for the future %%%
% \usepackage{parskip}
% \usepackage{textcomp}
% \usepackage{soul}

\begin{document}
\maketitle

% approval pdf

\tableofcontents

\listoffigures

\chapter*{Acknowledgments}
Acknowledged!

\chapter*{Abstract}
This research develops a new synthetic nervous system for control of joints using pneumatic actuators in order to create a more efficient and adaptable walking robot system. This design implements new features not previously seen in a control oriented nervous system. It develops 3 major design components not previously used for control of antagonistic pneumatic actuators. It uses an internal model of the actuators to esimate the state of the joint. It uses internal estimates of the dynamics of the joint to continually optimize the control output. Additionally, it updates its own internal system model with a memory-like loop to allow the controller to adapt to any joint and trajectory. The controller allows for the replacement of proportional or other control designs with a learning system that decreases wasted antagonistic muscle activation and wasted energy, decreases phase lag and increases trajectory tracking accuracy.

\chapter{Introduction}
\begin{itemize}
\item General Task
\item Robots for the general task
\item Specifications for task
\item Robotic Locomotion?
\end{itemize}
Walking (especially bipedal) is hard

Animals achieve stability and effective walking through redundancy (ex. 4 legs,
6 legs, etc) and multi-level neuron control via sensory feedback and descending
commands. Studies have shown that animals can still exhibit significant portions
of walking behavior without higher level control from the brain and, in certain
conditions, without sensory feedback.

Animals adapt to their environment through physical structure and compliance as
well as adapting through a learning process in the nervous system. While the
simulation or replication of both the physical and neural systems in their
entirety is a very complex task, lessons can be taken from the neural control of
muscles to better control a walking robot.

\section{Task Overview}

This thesis focuses on control of joints for the canine-inspired
Puppy robot in a walking scenario. The two distinguishing physical features of 
the robot are its pneumatic muscle actuators and a design that leans heavily
on the mechanics and skeleton of canines.

% TODO(buckbaskin): insert figure
Figure of Puppy

% TODO(buckbaskin): finish this thought
Pneumatic actuators are used because...

% TODO(buckbaskin): finish this thought
Canine design is used because...

\section{Why Walking?}

Walking robots, and more specifically quadruped walking robots, offer a number 
of advantages over wheeled robots. First, walking robots can cover much more 
varied terrain compared with wheeled vehicles. Second, quadruped and walking
robots with more than 4 legs offer inherent stability compared with biped robots
on the same terrain via redundancy and the ability to move one limb at a time
while keeping the center of mass over a supported area.


On the other hand, walking robots have disadvantages that mean they are not
necessarily the right option for every situation.

Increased energy usage (energy per meter metric?). This comes from the
requirement that the active control in the robot is required to support the
weight of the robot with motors or muscles instead of having weight statically
supported by the structure in the case of wheels.

% TODO(buckbaskin): write about this more
Lower maximum speed


One option that has been explored as an intermediate step are Whegs. These
designs allow for increased climbing ability on uneven terrain while retaining
many of the advantages of wheels including ease of control (although additional
% TODO(buckbaskin): 1-2 citations for Whegs
work must be done to synchronize a tripod gate) and increased speed.


However, canines and other legged animals have adapted to efficient and fast 
methods for walking using muscles that suggest robotic design inspired by 
biomechanics of a dog can offer insights into making a more efficient and 
faster robot that retains the advantages of legged robots for walking on uneven
% TODO(buckbaskin): incorporate a second reason for walking robots
terrain and remaining stable in varied environments.

\section{Specifications and Important Challenges}

For control of a walking robot, based on previous challenges

% TODO(buckbaskin): break this into subsections and a paragraph for each
% challenge
Challenges:
\begin{itemize}
\item Peak position during hip actuation isn't accurate (5-15 deg) 
\item Delays between sensor data off the robot -> lab view -> animatlab -> lab
view -> robot cause tracking problems
\item Scapula drives forward motion for front legs in the reverse Z orientation
\item Improve torque profiles to eliminate external support, even close to
maximum activation
\item Better internals and internal understanding of dynamics to match desired
and intended activations
\end{itemize}

% TODO(buckbaskin): break this into subsections and a paragraph for each
% requirement
Requirements:
\begin{itemize}
\item Increase position accuracy to less than 5 degrees from maximum desired position
\item Minimize phase shift between desired and achieved trajectory
\item Decrease torque errors between neuron desired model and executed torques
\end{itemize}

% TODO(buckbaskin): break this into subsections and a paragraph for each
% design goal
Internal Design Goals:
\begin{itemize}
\item Incorporate and ``understanding" of dynamics
\item Optimize current control for future position or, stated another way,
control for the current time/state with delayed sensory information
\end{itemize}

\section{This Work}

\Cref{chap:controller_design} discusses the design of a new type 
of controller that incorporates the
design goals of understanding the dynamics, projecting forward in time and
optimizing the controlled output based on the understanding of dynamics.

\Cref{chap:neuron_design} discusses the implementation of the controller within 
a neuron
simulation. This discussion talks about some simplifications and approximations
that were made in the process, as well as areas of the network that are more
complex improvements over the controller design in \Cref{chap:controller_design}.



\chapter{Literature Review/Background}
\begin{itemize}
\item General Robot Class
\item Robots for a more particular task
\item Heritage of a robot feature
\end{itemize}
\section{Pneumatic Actuators}

\cite{latexcompanion}

\section{Neuron Design}

\section{Controllers}

\chapter{Controller Design}
\begin{itemize}
  \item Simulation
  \item Design of each component x 5
\end{itemize}
% \chapter{Simulation Environments and Models}
The controller design had 3 major goals derived from
requirements and challenges faced with a neural implementation in
\cite{HuntPhDThesis}. See \myref{chap:introduction} for more details.

\begin{enumerate}
\item Increase position accuracy to less than 5 degrees from maximum desired
position
\item Minimize phase shift between desired and achieved trajectory
\item Decrease torque errors between neuron desired model and executed torques
\end{enumerate}

Requirement 2 is fulfilled in
the controller through sensor fusion to estimate state at the time of sensor
readings (assumed to be ``current" time) and then using an internal model of the
joint dynamics to predict the evolution of the joint's position and velocity
given a particular controlling torque. From there, an internal optimization loop
can run to identify a goal torque with sufficient resolution.

An accurate internal model will help meet all 3 requirements. In particular, an
internal model directly solves Requirement 3 by matching controlled torques to
pressures accounting for the pneumatic muscle dynamics. The internal model
designed into the control is designed to adapt to the actual observed
characteristics of the joint, allowing for changing dynamics over time (for
example, adding or removing weight from the robot, degrading pneumatic muscles
or changing positions of other limbs),
adapting to different joints with the same controller and increasing the
robustness of the controller to variations from the original estimated
properties of the robot.

Requirement 1 is met by a combination of the design decisions above and
constitutes a metric by which to measure the effectiveness of the new design
relative to old designs, regardless of implementation.

\section{Sensor Fusion}

Within the controller, state estimation is done in a relatively simplified
manner. This trades off accuracy for computation time, where the estimation was
refined to run sufficiently accurately that the rest of the system was robust to
small estimation errors.

\subsection{Pneumatic Actuator Simulation/Simplification}

The pneumatic actuators and their existing pressure controller were simulated 
at varying degrees of fidelity ...

% TODO(buckbaskin): finish this thought based on code

\subsection{Centered Divided Difference}

The velocity was estimated with a fixed window of the last 3 position readings.
From there, a centered divided difference formula was used to estimate the
slope and estimate velocity. This was found to be more accurate than a forward
divided difference method only incorporating two points in practice when run
against simulated data.

% TODO(buckbaskin): write the two equations

% TODO(buckbaskin): show a plot with the two different estimations side by side
% to show error

\section{Optimizing Torque Control}

Once the current state of the robot was determined, iteratively optimizing the
desired torque was relatively simple. First, bounds were selected from estimates
of the maximum and minimum (maximum negative) torque that can be applied to the
joint and a forward projection mechanism based on the internal physics model was
used to estimate the resulting position. From there, a binary search procedure
can be used to repeatedly shrink the bounds around the optimal torque. The final
step was to convert the desired torque into a pressure control.

\subsection{Forward Projection of State}

Given the state estimation, project forward.

Also, can use bounds on estimated state, say high and low velocity, to project
forward through multiple iterations, and optimize for the torque that, given the
bounds, is most likely to produce the best result. This work is unnecessary for
a well implemented state estimation system, but is helpful for increasing
robustness to potential sensor reading errors or estimation errors.

% TODO(buckbaskin): complete this thought

\subsection{Torque to Pressure}

% TODO(buckbaskin): complete these two steps
Show the math

Also, talk about designing the desired pressure around the knowledge of a
bang-bang controller under the hood for more accurate pressure control

\section{System Modeling}

Parameter weight updates

\subsection{Actual Error Calculation}

% TODO(buckbaskin): adapt existing work on how the error update should be done
This comes from a write up that I've already done

\subsection{Evaluation of asymmetric error concerns}o

% TODO(buckbaskin): use some sort of math to show that the over/underestimation
% of damping, etc. can lead to underdamped or overdamped conditions. Prefer
% overdamped control


\chapter{Neuron Design}
\begin{itemize}
  \item Design of each component x 5
\end{itemize}
% \chapter{Simulation Environments and Models}
The synthetic nervous system controller is developed with a <insert name here>
neuron model. The neurons are modeled with a resting potential of -60 mV and a
maximum potential of -40 mV for a range of 20 mV.

% TODO(buckbaskin): make the capitalization of the networks consistent

\bbs{Key Neurons and Synapses}

The neuron controller network is made up of a set of engineered synapses
designed to emulate arthimetic operations. These can be grouped into two
categories: excitatory neurons and inhibitory neurons. Most neurons were tuned
via an optimization process to identify the best equilibrium potential and
synaptic conductance. Some neurons were tuned by hand for specific behaviors in
a subsection of the overall controller.

\bbss{Excitatory Synapses}

Except where otherwise noted, a value of 134 mV was used for the equilibrium
potential. The pre-synaptic threshold is -60 mV and the pre-synaptic saturation
level is -40 mV.

\bbsss{Signal Transfer}

The signal transfer synapse is designed to pass the voltage of the input neuron
to the output neuron in the active range of the neuron. It is often used to add
the value of two or more neurons together in an output neuron.
It has a synaptic 
conductance of 0.115 microsiemens. 

\bbsss{Inverted Signal Transfer}

By combining two signal inversions, a more accurate signal transfer synapse was
created. This involves an extra neuron; however, it leads to a more precise
transfer of the voltage level of the input neuron to the output neuron. In
general, the synapse was implemented with two Signal Inverter (Stimulated)
synapses.

\bbsss{Signal Amplifier 2x}

The signal amplifier synapse is designed to pass the voltage of the input neuron
to the output neuron with a 2x gain. This was tuned with input values from 0 to
10 mV. Higher inputs saturate the output neuron. It has a synaptic conductance
of 0.23 microsiemens.

\bbsss{Signal Amplifier 4x}

The signal amplifier has the same function as the 2x amplifier but with a x4 
gain. This was tuned with input values from 0 to
5 mV. Higher inputs saturate the output neuron. It has a synaptic conductance
of 0.46 microsiemens.

\bbsss{Signal Reduction 0.2x}

The signal reduction synapse is designed to pass the voltage of the input neuron
to the output neuron in the active range of the neuron, but with a loss of 0.2x. 
It has a synaptic  conductance of 0.021 microsiemens. This neuron was tuned
over the complete range of input values, 0 to 20 mV.

\bbsss{Signal Reduction 0.5x}

The signal reduction synapse is designed to pass the voltage of the input neuron
to the output neuron in the active range of the neuron, but with a loss of 0.5x. 
It has a synaptic  conductance of 0.054 microsiemens. This neuron was tuned
over the complete range of input values, 0 to 20 mV.

\bbsss{Convert Forward Positive}

The convert forward positive synapse is designed to convert the representation
of a value in a single neuron, for example the neuron that represents the 
position of the joint, to the same value represented in two neurons. This
synapse converts the value when the value is above -50 mV to values between 0 
and 20 mV. The pre-synaptic
threshold is -50 mV and the synaptic conductance is 0.115 microsiemens.

\bbsss{Torque Pressure Converter}

The torque pressure converter synapse is designed to perform the torque-pressure
calculation approximation. It has a synaptic 
conductance of 0.048 microsiemens.

\bbss{Inhibitory Synapses}

Except where otherwise noted, the equilibrium potential of inhibitory neurons
is simulated as -100 mV. This is the lowest value possible in the simulation.
The pre-synaptic threshold is -60 mV. The pre-synaptic saturation level is -40
mV.

\bbsss{Signal Inverter}

The signal inverted synapse is designed to decrease the voltage of the output 
neuron proportional to the voltage of the input neuron. It has a synaptic 
conductance of 0.55 microsiemens.

\bbsss{Signal Inverter (Stimulated)}

The signal inverted synapse is designed to decrease the voltage of the output 
neuron proportional to the voltage of the input neuron. This synapse is tuned
slightly differently from the standard signal inverter because there was an
observed difference in voltage when a stimulus current was applied to the output
neuron along with other incoming synapses. It has a synaptic 
conductance of 0.5 microsiemens.

\bbsss{Signal Invert Reduction 0.2x}

The signal inverted reduction synapse is designed to decrease the voltage of
the output 
neuron at a 0.2x loss compared to the voltage of the input neuron. It has a 
synaptic conductance of 0.093 microsiemens.

\bbsss{Signal Invert Reduction 0.5x}

The signal inverted reduction synapse is designed to decrease the voltage of
the output 
neuron at a 0.5x loss compared to the voltage of the input neuron. It has a 
synaptic conductance of 0.22 microsiemens.

\bbsss{Signal Inverter Amplifier 2x}

The signal inverted amplifier synapse is designed to decrease the voltage of
the output 
neuron at a 2x gain to the voltage of the input neuron. It has a synaptic 
conductance of 1.11 microsiemens.

\bbsss{Signal Inverter Amplifier 4x}

The signal inverted amplifier synapse is designed to decrease the voltage of
the output 
neuron at a 4x gain to the voltage of the input neuron. It has a synaptic 
conductance of 2.3 microsiemens.

\bbsss{Integral Inhibitor}

The integral inhibitor synapse is designed to help both neurons maintain a 
stable value unless an external current is applied. The values for this
synapse are based on \cite{NickFunctionalSubnetwork}. The synapse has a  
conductance of 0.5 microsiemens.

\bbsss{Convert Forward Negative}

The convert forward negative synapse is designed to convert the representation
of a value in a single neuron, for example the neuron that represents the 
position of the joint, to the same value represented in two neurons. This
synapse converts the value when the value is below -50 mV to a positve value 
from 0 to 20 mV. The pre-synaptic
saturation is -50 mV and the synaptic conductance is 0.5 microsiemens.

\bbsss{Signal Divider}

The signal divider synapse is designed to reduce the effect of another input
synapse, where the reduction increases with increased input voltage to the
signal divider synapse. It is distinguished from the signal multiplier synapse
in that the value never reaches 0. Intuitively, this is a replication of 
division where dividing by a large number makes the quantity small but never 0.
It has a synaptic conductance of 20 microsiemens. The equilibrium potential of
the synapse is -60 mV (equal to the resting potential of the input and output
neurons).

\bbsss{Signal Multiplier}

The signal divider synapse is designed to reduce the effect of another input
synapse, where the reduction deccreases with increased input voltage to the
signal divider synapse. It is distinguished from the signal divider synapse
behavior in that the value reaches 0. Intuitively, this is a replication of 
multiplication where multiplying by 0 will make any value 0.
It has a synaptic conductance of 19.75 microsiemens. The equilibrium potential 
of the synapse is -61 mV.

\bbs{Sensor Fusion}

The Sensor Fusion neuron network performs essentially the same function as the
sensor fusion network in the prototype controller. In this case, 3 neurons
represent the 3 sensor inputs available in a joint: position (``Theta"),
extension muscle pressure (``Ext Pres") and flexion muscle pressure
(``Flx Pres"). The outputs for the network are the estimates for current 
position, velocity and acceleration.

\bbss{Velocity Fusion Network Components}

\bbsss{Differentiator Network}

% 2
% TODO(buckbaskin): talk about differentiator network
The velocity network is based on the Differentiator Network presented in 
\cite{NickFunctionalSubnetwork}.

% 3
% TODO(buckbaskin): Figure of differentiator network

\bbss{Velocity Fusion Network}

The velocity network is based on the Differentiator network. 
The one major change from the network,
as presented, is the inclusion of a second $U_{post}$ neuron to represent the
negative derivative of the position (negative velocity).

% 4
% TODO(buckbaskin): Figure of my velocity network

% TODO(buckbaskin): make sure my inhibitory and excitatory synapses are visualized properly

This represents
a common pattern used across the network where two neurons are used to represent
a single value. One neuron represents positive levels of the variable and is 
at or below resting potential when the value is negative. The other neuron is
above resting potential for negative values and is at or below resting potential
when the value is positive. 

The motivation for increasing the complexity of the
network (often making it more than twice as complicated) is to increase the
effective range of values that the neuron can represent at the same fidelity and
to increase the accuracy of zero. When a single neuron represents positive and
negative values of equal magnitude, the value of 0 is represented at 50 mV; 
however, after passing through a signal transfer synapse this value is often
slightly higher, up to 52 mV. This means that comparing the two neurons (the
original neuron and the signal transfer) yields a slightly positive error
instead of near zero error.

\bbss{Acceleration Fusion Network Components}

\bbsss{Absolute Value Network}

Within the acceleration fusion network, the absolute value of the position is
used. This is calculated by first splitting the position into its two neuron
representation. From there, the sum of the two neurons is used as the absolute
value. This takes advantage of the definition of each side of the two neuron
representation falling below resting potential when the other neuron is active.
This means the signal transfer synapse from the below zero neuron will have no
effect.

% 5
% TODO(buckbaskin): Figure of my absolute value network

\bbsss{Integration Network}

The integration network used throughout the neuron controller is based heavily
on the Integrator Network in \cite{NickFunctionalSubnetwork}. Two neurons are
designed to mutually inhibit each other so that the combined pair hold their
values. Individual neurons can be treated as a leaky integrator; however, their
voltage tails off over time if there is no maintenance current. The integration
network itself is tuned by changing the time constant of the component neurons
to adjust how much the voltage of the integrator network changes for an input
current.

\bbsss{Convert Torque to Pressure}

% 6
% TODO(buckbaskin): talk about how this approximation came about through simulation
Linearized approximation to help with neurons

% TODO(buckbaskin): regenerate the PressureTorque figure with axis labels
\begin{figure}[h!]
\centering
\includegraphics[width=5in]{neuron_design/FigPressureTorque}
\caption{Torque/Pressure relation observed in simulation}
\label{fig:PressureTorque}
\end{figure}

\bbsss{Pressure Estimation Loop}

Within the Acceleration Network, there is a feedback loop that is used to 
estimate the torque applied by a given pressure. First, the loop ``initializes"
with an extension torque guess from the integrator. Second, the extension
torque is converted into extension pressure. Third, the estimated extension
pressure is compared with the sensed pressure. If there is an delta between the
two, the extension torque guess is modified in turn and the cycle repeats.

% 7
% TODO(buckbaskin): Figure of pressure loops

This architecture is mirrored for flexion torque.

\bbss{Torque to Acceleration Network}

% 8
% TODO(buckbaskin): talk about T2A

\bbss{Acceleration Fusion Network}

The acceleration fusion network is a combination of an integrator, absolute
value network, the pressure estimation loop and a torque to acceleration
network. In total, the network uses a combination of smaller networks to
estimate the torque applied from sensed pressure and then combine the extension
and flexion torques together to get a net torque and acceleration.

\section{Optimizing Torque Control}

% 9
I/O summarizing, goals
position, velocity, desired position -> torque

\subsection{Small Networks}

% 10

\subsection{Mid Networks}

% 11

\subsection{Entire Network}

% 12

\section{Torque to Pressure}

% 13
I/O summarizing, goals
convert desired torque pressure in neurons

\subsection{Small Networks}

% 14

\subsection{Mid Networks}

% 15

\subsection{Entire Network}

% 16

\section{System Modeling}

% 17
I/O summarizing, goals
Parameter weight updates

\subsection{Small Networks}

% 18

\subsection{Mid Networks}

% 19

\subsection{Entire Network}

% 20

\bbs{Combined Controller Network}

\begin{figure}[h!]
\centering
\includegraphics[width=5in]{neuron_design/NetworkLayout}
\caption{Overview of major network components}
\label{fig:NetworkLayout}
\end{figure}
% TODO(buckbaskin): redo this with the complete actual network

\chapter{Testing/Methods}

\chapter{Results}

\chapter{Conclusions and Future Work}
\bbs{Conclusion}

This thesis presents the design of a new style of biologically inspired
controller for controlling revolute joints with antagonistic pneumatic
artificial muscles. First, the literature surrounding the characterization of
the actuators and the methods used to control them is discussed in
\myref{chap:lit_review}. Second, the
design of a prototype for a controller is discussed in
\myref{chap:controller_design}. Third, the implementation of the design in a
synthetic nervous system is discussed in \myref{chap:neuron_design}. The methods
for testing the controller are discussed in \myref{chap:methods}. The results of
the tests are presented in \myref{chap:results}.

The controller implements new features not previously used in synthetic nervous
system design and not previously used for control of joints actuated by
pneumatic artificial muscles. The system has a whole has been shown to offer
increased accuracy in joint position and decreased phase shift. This was
combined with a better internal model of the actuators themselves to more
accurately model the forces and torques applied in order to continue successful
operation even near the maximum output of the actuators. On the other hand, the
successful operation of the controller is dependent on the tuning of a number of
internal parameters that are sensitive to slight overestimates that lead to
oscillation and loss of control of the joint. This has been mitigated by
asymmetric model updates that favor stable error and starting from known stable
estimates; however, this leads to tracking that does not meet the accuracy goals
discussed in the design when the parameters are too conservative.

The implementation of the controller in a synthetic nervous system led to the
design of unique nervous system functional subnetworks; however, some networks
were found to be inconsistent with their standard controller output, leading to
potential deficiencies in observed performance. This would suggest that a more
bionic approach that leverages mathematical models directly for complicated
calculations, such as the relation between pressure and torque for pneumatic
artificial muscles, and the dynamic benefits of a synthetic nervous system for
aspects of the controller that require less precision and benefit from insights
gained from a deeper biological understanding of how animal nervous systems
control joints.

\bbs{Future Work}

The next step for the design of the controller is experimental testing to
characterize performance on actual hardware. During simulation and testing,
many aspects of the controller worked well; on the other hand, some did not
behave as well as expected. On hardware, there is always some variation that may
show that the design choices made were effective and practical solutions.

There is also some potential for improvement of the design of the controller.
Further analysis of the stability of the controller may lead to a better model
update procedure. In particular, algorithms such as an Extended Kalman Filter
are used in many applications across robotics for estimating parameters from
sensor data, but the Kalman Filter algorithm itself was not used because it does
not have an effective way to be implemented within a synthetic nervous system.
New biological research % TODO(buckbaskin): cite that paper with the toroid structure
suggests that animals have groups of neurons that estimate orientation and
% TODO(buckbaskin): and that time and space cells paper thing I found
position in time and space. This research suggests potential approximations for
spatial estimation that may be useful for estimation of other parameters to
emulate or replace the use of a Kalman Filter.

Another area where the neuron controller has room for improvement is
implementation of certain subnetworks that don't effectively implement their
counterparts in the prototype. There were approximations made to match neuron
and synapse behavior at a low level to some of the mathematics; however, taking
a larger system approach with an aim to design neuron connections to emulate
behavior of larger pieces of the system made lead to a more successful design.
The end goal is to design a working controller with similar properties, this
doesn't need to be achieved by direct copying. The testing methods implemented
in this thesis offer a good system for iterating on this kind of design. Test
inputs can be set up once and then higher level component behavior can be
compared quickly and visually.

Overall, the design of the controller presents a new style of design that can be
used to create optimal controllers for actuation systems that are highly
non-linear and tend to perform poorly when simpler controller designs are used.
There is room for improvement, but the methods used in the thesis offer a
framework for improvement and determining the areas with the largest impact.


\appendix
\chapter{Network Topologies}
The code and configuration files for this project are available on Github at \url{https://github.com/buckbaskin/animatlab}. In general, each section corresponds to a directory in the repository (the title links to the folder).

\bbs{\github{PuppyNeuronPlayground}}

This folder contains the Animatlab project used to test and design the synthetic nervous system controller. It includes a number of data tools for testing and development.

\bbs{\github{stability}}

This folder contains most of the Python simulation resources used for the project. It was originally named for its intention to measure and visualize the stability of controllers, but it also contains other work as well. Key files are \texttt{constant\_pressure.py}, \texttt{max\_torque.py}, \texttt{pressure\_torque.py}, \texttt{reduced\_controller.py}, \texttt{simple\_mass\_model.py}, \texttt{simulation.py}, and \texttt{torque\_projection.py}.

\bbs{\github{fitting\_neurons}}

This folder contains Python scripts used to fit the neuron model in \cite{NickFunctionalSubnetwork} to arithmetic operations.

\bbs{\github{testing}}

This folder was used to plan out testing setups for use in the controller tests.

\bbs{\github{writeup}}

This folder contains the materials used to create the full thesis writeup as well as a conference paper.

\bbss{\github{writeup/data}}

This subfolder contains the raw data exports from Animatlab as well as the Python script to convert the data to a more common \texttt{.csv} format and the converted CSV files.

\bbss{\github{writeup/scripts}}

This subfolder contains scripts used to more consistently visualize Animatlab data using \texttt{matplotlib}, a Python library.

\bbs{\github{Poster}}

This folder contains the materials used to create a poster for Case Western's Research Showcase.

\bbs{\github{ICanDoMath}}

This was a small Animatlab project used to verify the tuning of synapses in Python. Some of this functionality was ported to the main design project.

\bbs{\github{docs}}

This folder contains notes from eariler in the project.

\printbibliography

\end{document}
