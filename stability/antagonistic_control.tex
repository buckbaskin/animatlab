\documentclass[12pt, letterpaper, oneside, notitlepage, onecolumn]{article}
\author{William Baskin}
\title{Controller for Large Angle Change with Antagonistic Muscle-Like Actuators}
\pagestyle{plain}

\usepackage{parskip}

\usepackage{textcomp}
\usepackage[utf8]{inputenc}
\usepackage[english]{babel}
\usepackage{listings}
\usepackage{color}
\usepackage{verbatim}
% \usepackage{soul}
\usepackage[margin=0.69in]{geometry}

% math
\usepackage{amsmath, amssymb, amsthm}
% \usepackage{amsmath, amssymb, amsthm, gensymb}

\usepackage{graphicx}
\graphicspath{ {graphics/} }
% \includegraphics[height=6.75in,angle=270]{HW25}

\definecolor{dkgreen}{rgb}{0,0.6,0}
\definecolor{gray}{rgb}{0.5,0.5,0.5}
\definecolor{mauve}{rgb}{0.58,0,0.82}

\lstset{frame=tb,
  language=Python,
  aboveskip=3mm,
  belowskip=3mm,
  showstringspaces=false,
  columns=flexible,
  basicstyle={\small\ttfamily},
  numbers=none,
  numberstyle=\tiny\color{gray},
  keywordstyle=\color{blue},
  commentstyle=\color{dkgreen},
  stringstyle=\color{mauve},
  breaklines=true,
  breakatwhitespace=true,
  tabsize=3
}

\DeclareMathOperator*{\argmax}{arg\,max}

\DeclareMathOperator*{\argmin}{arg\,min}

\newcommand{\subsubsubsection}{\paragraph}
\newcommand{\bbs}[1]{\section{#1}}
\newcommand{\bbss}[1]{\subsection{#1}}
\newcommand{\bbsss}[1]{\subsubsection{#1}}
\newcommand{\bbssss}[1]{\subsubsubsection{#1}}

\newcommand{\norm}[1]{\left\lVert#1\right\rVert}

\usepackage[pdftex,
    pdfusetitle
    ]{hyperref}

\begin{document}
\maketitle

\bbs{Controller Model}

\bbss{Control Problem}

Trajectory tracking posed as a time series of desired positions.

\bbsss{Inputs}

A stream of desired positions. Ex. a sin wave, a step function, CPG oscillating
pattern

Input effects to consider: Rate of change, systematic error in tracking for a
certain desired velocity relative to the stiffness of the controller. Also, the
similarity of the test inputs to actual desired motion of the joint.

\bbsss{Sensors}

Each joint has a position sensor and pressure sensors for each actuator. The
robot feet also have pressure/contact sensors. 

Sensor effects to consider: The pressure sensors are
physically filtered to provide a less erratic but slightly delayed.

\bbsss{Output}

The controller takes in the desired position and sensor data and outputs desired
pressures for two actuators. This comes from taking the desired position and 
sensor data, calculating a desired acceleration for the joint 
$\ddot{\theta}(\theta_{err}, \dot{\theta}_{err})$. This is a function of 
controller meta-parameters (ex. stiffness) and the derived gains for 
positional error and velocity error.

The controller outputs desired pressures. This is one step different from the
controller as implemented now that takes in desired position and holding torque
and sets the actuators to hold equal and opposite torques of the same (given)
value.

\bbsss{Controller Meta-Parameters}

\begin{itemize}
\item Stiffness
\item Proportional Gain for Desired Torque
\item Derivative Gain for Desired Torque
\item Torque limits
\end{itemize}

These parameters are "pre-"determined by the controller and determine the
specifics of how the controller tracks the input desired positions. Due to the
actuation schema of the robot (two antagonistic pneumatic muscles), the joint
position, torque and stiffness can be controlled some combination of
independently. This means that the controller gains for position/velocity/other
potential terms can be updated over time.

Small changes in joint position are controlled by changing the desired pressure
to the calculate pressure for the point and equal and opposite actuator torques.
This results in the actuators behaving like a proportional spring around the
desired joint angle. The actuators are also assumed to have some inherent
damping properties (hysteresis? (s.p.)) that makes an otherwise linear spring
stable in small changes. This damping may or may not have significant effects
for large motions that are at this time not yet analyzed.

Large changes are calculated for combinations of error in joint position and
joint velocity where the desired pressure change is large (larger than the
tolerance for the bang-bang pressure controller). The desired torque is
calculated from the determined PD control gains. Then the pressures for each
actuator are determined from the desired stiffness and desired torque at the
current joint position.

Controller effects to consider: bang-bang pressure updates, low hz update for
the controller causing instability at high gains, minimum gains deteremined by
tracking performance definitions, maximum gains determined by controller rate or
other effects.

Further considerations: Can an outside observer detect these "failure modes"?
And if so, are they recoverable by changes in controller parameters. The
controller could also be given the desired velocity of the trajectory to help
with tracking. The current assumption is desired 0 velocity at the given point.

\bbss{Mass/Dynamics Model}

Each joint of the robot is doing control of a limb that will be (for now)
treated as a pendulum with a mass at the center of the limb. This provides
approximations for rotational inertia around the joint and systematic forces,
ex. gravity. 

Each actuator also has damping properties. For small joint angle changes where
the change in volume of the actuator is small, the muscles experience some
internal friction or other effects that scrub energy from the system. For large
changes in volume, the friction effects of pushing the air volumes through small
feeder tubes will be modeled to estimate the damping from non-zero fill time and
energy loss.

\bbss{Pressure Model}

The pressure model is based on work done by Alex Hunt that models the pressure
as a function of strain and desired force.

\bbss{Forward Pressure Model}

Based on desired controller parameters, how is pressure calculated?

\begin{equation}
P_{actuator} = P(\theta_{des}, \tau_{des}) = P(\kappa_{des}, F_{des})
\end{equation}

The desired position and torque are converted into a desired strain and force.
The position determines the normalized length of the actuator relative to the 
relaxed length and the angle at which the actuator applies a torque. This
determines the required force to exert the torque.

\bbss{Inverse Pressure Model}

Based on linearized pressure model for two antagonistic actuators, desired 
change in pressure, what is the torque on the joint?

\bbs{Linear Stability for Small Joint Angles}

This work was previously done, and shows that the joint behaves as a
proportional torsion spring when the correct pressures are set for the desired
joint angle and holding torque. The damping effects of the actuators themselves
are currently assumed to lead to joint stability in this case.

\bbs{Linear Stability for Large Joint Angles}

For large joint angles, a proportional-derivative control layer is added on top
of the pressure calculations for a given joint torque to control the joint to a
desired angle.

\bbss{Model Parameters and Terms}

\begin{itemize}
\item $\theta_{des}$ Desired Joint Position
\item $\theta$ Current Joint Position
\item $\dot{\theta}$ Current Joint Velocity
\item $\tau$ Current Joint Torque
\item $\tau_{des}$ Desired Joint Torque
\item $\tau_{max}$ Maximum Joint Torque Limit
\item $\tau_{min}$ Minimum Joint Torque Limit
\item $P_{+}$ Pressure of the defined-positive direction actuator (contract
increases joint angle)
\item $P_{-}$ Pressure of the negative actuator

\item $\mathcal{S}_{des}$ Desired Stiffness of the joint
\item $\mathcal{K}_{p}$ Proportional Gain
\item $\mathcal{K}_{v}$ Velocity Gain

\item $\mathcal{J}$ Moment of Inertia
\item $\mathcal{M}$ Mass of Joint
\item $C_{I}$ Internal Damping of Actuator (based on position change)
\item $C_{G}$ Gaseous Damping of Actuator in motion (based on volume change)
\end{itemize}

\bbss{High Level Model}

\begin{equation}
\theta_{err} = \theta - \theta_{des},
\dot{\theta}_{err} = \dot{\theta} - \dot{\theta}_{des}
\end{equation}

\begin{equation}
\tau_{des} = -\mathcal{K}_{p} \theta_{err} + -\mathcal{K}_{v} \dot{\theta}_{err}
\end{equation}

% TODO(buckbaskin): Can I write this in a more elegant way?
\begin{equation}
\tau_{des, +} = max(\tau_{des}, 0) + \mathcal{S}_{des}
\end{equation}

\begin{equation}
\tau_{des, -} = min(\tau_{des}, 0) + \mathcal{S}_{des}
\end{equation}

\begin{equation}
P_{+} = P_{+}(\theta, \tau_{des, +})
\end{equation}

\begin{equation}
P_{-} = P_{-}(\theta, \tau_{des, -})
\end{equation}

The pressures are calculated from the current position ($\theta$) and the
desired torque at this point. The pressure calculations are equivalent but with
different constants representing the mirrored geometry.

\bbss{Dynamics}

Assuming the pressure update and sensor measurement are continuous and the joint
exactly fits the model, the controller can be thought of as directly controlling
the torque on the joint. 

For some joint, the above assumptions and the mass model, the behavior of the
joint can be described as follows:

\begin{equation}
M(\theta) \ddot{\theta} + C(\theta, \dot{\theta}) \dot{\theta} + N(\theta) \theta = 
\tau_{controller}
\end{equation}

$\mathcal{M}$ is the moment of inertia term. $\mathcal{C}$ is the combined term
of $\mathcal{C}_{I}$ and $\mathcal{C}_{G}$ based on the current position and
velocity (which determines $\frac{\partial V}{\partial t}$ for each actuator's
$\mathcal{C}_{G}$ term). $N$ is the conservative forces (ex. gravity) acting on
the limb. The $\tau$ in this case is exactly the desired torque for the current
state as calculated as $\tau_{des}$ in the controller. Substituting for the
desired torque:

\begin{equation}
M(\theta) \ddot{\theta} + C(\theta, \dot{\theta}) \dot{\theta} + N(\theta) \theta = 
\tau_{des} = -\mathcal{K}_{p} \theta_{err} + -\mathcal{K}_{v} \dot{\theta}_{err}
\end{equation}

Assuming the joint is moving relative to a reference at $\theta_{des}$ and a
velocity of $\dot{\theta}_{des}$ at that point ($\dot{\theta}_{des}$):

\begin{equation}
M(\theta) \ddot{\theta}
+ C(\theta, \dot{\theta}) (\dot{\theta} - \dot{\theta}_{des}) 
+ N(\theta)(\theta - \theta_{des}) = 
-\mathcal{K}_{p} (\theta - \theta_{des})
+ -\mathcal{K}_{v} (\dot{\theta} - \dot{\theta}_{des})
\end{equation}

Collecting terms

\begin{equation}
M(\theta) \ddot{\theta}
= 
-(C(\theta, \dot{\theta})
+ \mathcal{K}_{v}) (\dot{\theta} - \dot{\theta}_{des})
- (N(\theta)
+ \mathcal{K}_{p}) (\theta - \theta_{des})
\end{equation}

\begin{equation}
\ddot{\theta}
= 
-\dfrac{(C(\theta, \dot{\theta})
+ \mathcal{K}_{v})}{M(\theta)}
(\dot{\theta} - \dot{\theta}_{des})
-\dfrac{(N(\theta)
+ \mathcal{K}_{p})}{M(\theta)} (\theta - \theta_{des})
\end{equation}

\begin{equation}
\ddot{\theta}
= 
-\mathcal{K}_{v}^{*}(\theta, \dot{\theta})
\dot{\theta}_{err}
-\mathcal{K}_{p}^{*}(\theta)
\theta_{err}
\end{equation}

The acceleration is generally negative in sign compared with errors. When there
is 0 positional error, there is acceleration to decrease the velocity error.
When there is 0 velocity error, there is acceleration in the direction to
correct the positional error.

The acceleration is 0 when the two proportional terms cancel each other out:

\begin{equation}
0
= 
-\mathcal{K}_{v}^{*}(\theta, \dot{\theta})
\dot{\theta}_{err}
-\mathcal{K}_{p}^{*}(\theta)
\theta_{err}
\end{equation}

\begin{equation}
\dot{\theta}_{err}
= 
-\dfrac{\mathcal{K}_{p}^{*}(\theta)}{\mathcal{K}_{v}^{*}(\theta, \dot{\theta})}
\theta_{err}
\end{equation}

This indicates that, in this situation, if there is a positive velocity error, 
it occurs when there is a negative position error or vice versa. This means that
the control scheme will, as a result at a future time step, have a less negative
position error and the same velocity error. Then the acceleration will be
negative because the proportional term will have decreased leaving a larger
negative velocity correction term.

This control also decreases the energy with respect to the reference state
(desired position and velocity). Assuming the mass is at the center of mass and
the joint is similar to a pendulum:

\begin{equation}
\dfrac{1}{2}\mathcal{I}(\dot{\theta}^{2} - \dot{\theta}_{des}^{2})
+ ( -Mglcos(\theta) + Mglcos(\theta_{des}))
\end{equation}

\bbs{Sensitivity to Parameters}

Bang-bang, controller hz, mass estimate, damping estimate, as a function of gain

\end{document}
