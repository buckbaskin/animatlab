\documentclass[12pt, letterpaper, oneside, notitlepage, onecolumn]{article}
\author{William Baskin}
\title{Festo Model}
\pagestyle{plain}

\usepackage{parskip}

\usepackage{textcomp}
\usepackage[utf8]{inputenc}
\usepackage[english]{babel}
\usepackage{listings}
\usepackage{color}
\usepackage{verbatim}
% \usepackage{soul}
\usepackage[margin=0.69in]{geometry}

% math
\usepackage{amsmath, amssymb, amsthm}
% \usepackage{amsmath, amssymb, amsthm, gensymb}

\usepackage{graphicx}
\graphicspath{ {a1/e1/} }
% \includegraphics[height=6.75in,angle=270]{HW25}

\definecolor{dkgreen}{rgb}{0,0.6,0}
\definecolor{gray}{rgb}{0.5,0.5,0.5}
\definecolor{mauve}{rgb}{0.58,0,0.82}

\lstset{frame=tb,
  language=Python,
  aboveskip=3mm,
  belowskip=3mm,
  showstringspaces=false,
  columns=flexible,
  basicstyle={\small\ttfamily},
  numbers=none,
  numberstyle=\tiny\color{gray},
  keywordstyle=\color{blue},
  commentstyle=\color{dkgreen},
  stringstyle=\color{mauve},
  breaklines=true,
  breakatwhitespace=true,
  tabsize=3
}

\DeclareMathOperator*{\argmax}{arg\,max}

\DeclareMathOperator*{\argmin}{arg\,min}

\newcommand{\subsubsubsection}{\paragraph}
\newcommand{\bbs}[1]{\section{#1}}
\newcommand{\bbss}[1]{\subsection{#1}}
\newcommand{\bbsss}[1]{\subsubsection{#1}}
\newcommand{\bbssss}[1]{\subsubsubsection{#1}}

\newcommand{\norm}[1]{\left\lVert#1\right\rVert}

\usepackage{hyperref}

\begin{document}
\maketitle

\section{$L_{angle}$}
\begin{equation}
l_{0} + l_{1} \cos{\left (\alpha + \theta_{des} \right )}
\end{equation}

\section{$k$}
\begin{equation}
\frac{1}{l_{rest}} \left(- l_{0} - l_{1} \cos{\left (\alpha + \theta_{des}
\right )} + l_{rest}\right)
\end{equation}

\section{$F$}
\begin{equation}
\frac{Torque_{des}}{d \cos{\left (\beta + \theta_{des} \right )}}
\end{equation}

\section{Pressure Model}
\begin{equation}
S a_{6} + \frac{Torque_{des} a_{5}}{d \cos{\left (\beta + \theta_{des} \right
)}} + a_{0} + a_{1} \tan{\left (a_{2} \left(a_{3} + \frac{- l_{0} - l_{1}
\cos{\left (\alpha + \theta_{des} \right )} + l_{rest}}{l_{rest}
\left(\frac{Torque_{des} a_{4}}{d \cos{\left (\beta + \theta_{des} \right )}} +
k_{max}\right)}\right) \right )}
\end{equation}

\section{Empirical Pressure Model (subs a*)}
\begin{equation}
15.6 S + \frac{1.23 Torque_{des}}{d \cos{\left (\beta + \theta_{des} \right )}}
- 192.0 \tan{\left (0.9342165 - \frac{2.0265 \left(- l_{0} - l_{1} \cos{\left
  (\alpha + \theta_{des} \right )} + l_{rest}\right)}{l_{rest} \left(-
  \frac{0.331 Torque_{des}}{d \cos{\left (\beta + \theta_{des} \right )}} +
  k_{max}\right)} \right )} + 254.3
  \end{equation}

\section{$\dfrac{\partial P}{\partial \theta}$}
\begin{equation}
\frac{1.23 Torque_{des} \sin{\left (\beta + \theta_{des} \right )}}{d
\cos^{2}{\left (\beta + \theta_{des} \right )}} - 192.0 \left(-
\frac{0.6707715 Torque_{des} \left(- l_{0} - l_{1} \cos{\left (\alpha +
\theta_{des} \right )} + l_{rest}\right) \sin{\left (\beta + \theta_{des}
\right )}}{d l_{rest} \left(- \frac{0.331 Torque_{des}}{d \cos{\left (\beta +
\theta_{des} \right )}} + k_{max}\right)^{2} \cos^{2}{\left (\beta +
\theta_{des} \right )}} - \frac{2.0265 l_{1} \sin{\left (\alpha + \theta_{des}
\right )}}{l_{rest} \left(- \frac{0.331 Torque_{des}}{d \cos{\left (\beta +
\theta_{des} \right )}} + k_{max}\right)}\right) \left(\tan^{2}{\left
(0.9342165 - \frac{2.0265 \left(- l_{0} - l_{1} \cos{\left (\alpha +
\theta_{des} \right )} + l_{rest}\right)}{l_{rest} \left(- \frac{0.331
Torque_{des}}{d \cos{\left (\beta + \theta_{des} \right )}} + k_{max}\right)}
\right )} + 1\right)
\end{equation}

\section{Linearization}

For a desired position of $\theta_{des}$ at 0 velocity, the linearization of the system is:

$f(x, y) = f(\theta_{des}, 0) + \dfrac{\partial f}{\partial
\theta}(\theta_{des}, 0) * (\theta - \theta_{des}) + \dfrac{\partial f}{\partial
\dot{\theta}}(\theta_{des}, 0) * (\dot{\theta} - 0)$

The partial fractions should get simpler if a particular joint geometry is used.
Right now they feel like a bit of a mess. Of note for the current system, there
is a different linearlization for expansion or contraction.

Rewriting the terms to highlight $\theta$ and $\dot{\theta}$:

$f(x, y) = f(\theta_{des}, 0) + \dfrac{\partial f}{\partial
\theta}(\theta_{des}, 0) * (\theta - \theta_{des}) + \dfrac{\partial f}{\partial
\dot{\theta}}(\theta_{des}, 0) * (\dot{\theta} - 0)$

$f(x, y) = \dfrac{\partial f}{\partial \theta}(\theta_{des}, 0) * \theta +
\dfrac{\partial f}{\partial \dot{\theta}}(\theta_{des}, 0) * \dot{\theta} + ()$

\section{But What do I linearlize}

\begin{equation}
Torque = \tau(\theta_{des}, \tau_{des}, \theta) = \tau(P(\theta_{des},
\tau_{des}), \theta)
\end{equation}

Where $\tau$ is the actual torque, $\theta_{des}$ is the desired static angle, 
$\tau_{des}$ is the desired holding torque, $\theta$ is the current angle and
$P$ is the calculated pressure for the joint based on the desired angle and
torque.

\begin{equation}
\tau = F * d cos(\beta + \theta)
\end{equation}

This is based on the geometry. $\theta$ here is the current angle. $d, \beta$
are constants that describe the geometry. So, what is F?

\subsection{Simplified Pressure Model}

In the static case, there should be minimal hysteresis, so the S term from the
previous model will be disregarded for the moment.

\begin{equation}
P = a_{0} + a_{1} * tan(a_{2} * (\dfrac{k}{a_{4} * F + k_{max}} + a_{3})) + a_{5} * F
\end{equation}

Collecting for F

\begin{equation}
\dfrac{P - a_{0}}{a_{1}} = tan(a_{2} * (\dfrac{k}{a_{4} * F + k_{max}} + a_{3})) +
\dfrac{a_{5}}{a_{1}} * F
\end{equation}

So, I don't think there's a neat and tidy way to solve for F. However, if my end
goal is to solve for torque and theta, then I could do a linear approximation to
estimate the spring-like properties. If I know the derivative of the pressure
function with respect to $\theta$ and $\tau$, then I can approximate the
relationship between the two for constant pressure near the approximation point.

\begin{equation}
P(\theta, \tau) \approx P(\theta^{*}, \tau^{*}) + '
\dfrac{\partial P}{\partial \theta}(\theta^{*}, \tau^{*}) * (\theta -\theta^{*}) +
\dfrac{\partial P}{\partial \tau}(\theta^{*}, \tau^{*}) * (\tau - \tau^{*})
\end{equation}

In particular, the approximation point is the desired static point:

\begin{equation}
P(\theta, \tau) \approx P(\theta_{des}, \tau_{des}) + 
\dfrac{\partial P}{\partial \theta}(\theta_{des}, \tau_{des}) * (\theta -\theta_{des}) +
\dfrac{\partial P}{\partial \tau}(\theta_{des}, \tau_{des}) * (\tau - \tau_{des})
\end{equation}

\begin{equation}
P(\theta, \tau) \approx P(\theta_{des}, \tau_{des}) + 
\dfrac{\partial P}{\partial \theta}(\theta_{des}, \tau_{des}) * (e_{\theta}) +
\dfrac{\partial P}{\partial \tau}(\theta_{des}, \tau_{des}) * (e_{\tau})
\end{equation}

This approximates the joint behaving as a torsion spring for angles near the
desired angle, with a resting angle of 0 that corresponds to the desired angle.
If this is rearranged to move constant terms to the left:

\begin{equation}
P(\theta, \tau) - P(\theta_{des}, \tau_{des})
\approx
\dfrac{\partial P}{\partial \theta}(\theta_{des}, \tau_{des}) * e_{\theta}
+ \dfrac{\partial P}{\partial \tau}(\theta_{des}, \tau_{des}) * e_{\tau}
\end{equation}

The assumed current control sequence is that the controller maintains constant
pressure. Therefore, the constant controlled pressure $P(\theta, \tau) = 
P(\theta_{des}, \tau_{des})$. This simplifies the expression to the following:

\begin{equation}
0
\approx
\dfrac{\partial P}{\partial \theta}(\theta_{des}, \tau_{des}) * e_{\theta}
+ \dfrac{\partial P}{\partial \tau}(\theta_{des}, \tau_{des}) * e_{\tau}
\end{equation}

Solving for torque:

\begin{equation}
e_{\tau}
\approx
- \dfrac{\dfrac{\partial P}{\partial \theta}(\theta_{des}, \tau_{des})}{
\dfrac{\partial P}{\partial \tau}(\theta_{des}, \tau_{des})}
e_{\theta}
\end{equation}

This would indicate that the actuators behave like a torsion spring ($\Delta \tau =
-\kappa \Delta \theta$) for small changes in angle of the joint. The spring constant $\kappa$:

\begin{equation}
\kappa
=
- \dfrac{\dfrac{\partial P}{\partial \theta}(\theta_{des}, \tau_{des})}{
\dfrac{\partial P}{\partial \tau}(\theta_{des}, \tau_{des})}
\end{equation}

This means that the stability of the joint, for small angles near the desired
angle, approximates proportional control if constant pressure is maintained.

\section{Conclusion and Future Work}

This analysis avoids two key elements of the joints. First, there are two
antagonistic actuators acting on the joint. This doesn't necessarily void the
analysis because the two actuators would both behave as springs with aligned
proportional restoring torques, so the effect would approximate two parallel
springs acting on the joint.

Second, the actuators are controlled so there is some constant torque acting on
the joint from opposing actuators. If it is assumed that the torques exactly
offset, as they are expected to do in the static stable case, then this analysis
is not invalidated; however, the derivatives w.r.t. pressure are on two
nonlinear functions that may not have the same behavior, especially because the
change in joint geometry is mirrored for two opposing actuators (ex. the change
in length of one for a given joint rotation is not necessarily the same and the
ability for the actuator to provide force/torque is dependent on its length).

Therefore, a more careful analysis of two antagonistic actuators acting on a
single joint will be performed in another document.

\qed

\end{document}
